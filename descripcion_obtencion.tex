\documentclass[]{article}
\usepackage{lmodern}
\usepackage{amssymb,amsmath}
\usepackage{ifxetex,ifluatex}
\usepackage{fixltx2e} % provides \textsubscript
\ifnum 0\ifxetex 1\fi\ifluatex 1\fi=0 % if pdftex
  \usepackage[T1]{fontenc}
  \usepackage[utf8]{inputenc}
\else % if luatex or xelatex
  \ifxetex
    \usepackage{mathspec}
  \else
    \usepackage{fontspec}
  \fi
  \defaultfontfeatures{Ligatures=TeX,Scale=MatchLowercase}
\fi
% use upquote if available, for straight quotes in verbatim environments
\IfFileExists{upquote.sty}{\usepackage{upquote}}{}
% use microtype if available
\IfFileExists{microtype.sty}{%
\usepackage{microtype}
\UseMicrotypeSet[protrusion]{basicmath} % disable protrusion for tt fonts
}{}
\usepackage[margin=1in]{geometry}
\usepackage{hyperref}
\hypersetup{unicode=true,
            pdfborder={0 0 0},
            breaklinks=true}
\urlstyle{same}  % don't use monospace font for urls
\usepackage{color}
\usepackage{fancyvrb}
\newcommand{\VerbBar}{|}
\newcommand{\VERB}{\Verb[commandchars=\\\{\}]}
\DefineVerbatimEnvironment{Highlighting}{Verbatim}{commandchars=\\\{\}}
% Add ',fontsize=\small' for more characters per line
\usepackage{framed}
\definecolor{shadecolor}{RGB}{248,248,248}
\newenvironment{Shaded}{\begin{snugshade}}{\end{snugshade}}
\newcommand{\AlertTok}[1]{\textcolor[rgb]{0.94,0.16,0.16}{#1}}
\newcommand{\AnnotationTok}[1]{\textcolor[rgb]{0.56,0.35,0.01}{\textbf{\textit{#1}}}}
\newcommand{\AttributeTok}[1]{\textcolor[rgb]{0.77,0.63,0.00}{#1}}
\newcommand{\BaseNTok}[1]{\textcolor[rgb]{0.00,0.00,0.81}{#1}}
\newcommand{\BuiltInTok}[1]{#1}
\newcommand{\CharTok}[1]{\textcolor[rgb]{0.31,0.60,0.02}{#1}}
\newcommand{\CommentTok}[1]{\textcolor[rgb]{0.56,0.35,0.01}{\textit{#1}}}
\newcommand{\CommentVarTok}[1]{\textcolor[rgb]{0.56,0.35,0.01}{\textbf{\textit{#1}}}}
\newcommand{\ConstantTok}[1]{\textcolor[rgb]{0.00,0.00,0.00}{#1}}
\newcommand{\ControlFlowTok}[1]{\textcolor[rgb]{0.13,0.29,0.53}{\textbf{#1}}}
\newcommand{\DataTypeTok}[1]{\textcolor[rgb]{0.13,0.29,0.53}{#1}}
\newcommand{\DecValTok}[1]{\textcolor[rgb]{0.00,0.00,0.81}{#1}}
\newcommand{\DocumentationTok}[1]{\textcolor[rgb]{0.56,0.35,0.01}{\textbf{\textit{#1}}}}
\newcommand{\ErrorTok}[1]{\textcolor[rgb]{0.64,0.00,0.00}{\textbf{#1}}}
\newcommand{\ExtensionTok}[1]{#1}
\newcommand{\FloatTok}[1]{\textcolor[rgb]{0.00,0.00,0.81}{#1}}
\newcommand{\FunctionTok}[1]{\textcolor[rgb]{0.00,0.00,0.00}{#1}}
\newcommand{\ImportTok}[1]{#1}
\newcommand{\InformationTok}[1]{\textcolor[rgb]{0.56,0.35,0.01}{\textbf{\textit{#1}}}}
\newcommand{\KeywordTok}[1]{\textcolor[rgb]{0.13,0.29,0.53}{\textbf{#1}}}
\newcommand{\NormalTok}[1]{#1}
\newcommand{\OperatorTok}[1]{\textcolor[rgb]{0.81,0.36,0.00}{\textbf{#1}}}
\newcommand{\OtherTok}[1]{\textcolor[rgb]{0.56,0.35,0.01}{#1}}
\newcommand{\PreprocessorTok}[1]{\textcolor[rgb]{0.56,0.35,0.01}{\textit{#1}}}
\newcommand{\RegionMarkerTok}[1]{#1}
\newcommand{\SpecialCharTok}[1]{\textcolor[rgb]{0.00,0.00,0.00}{#1}}
\newcommand{\SpecialStringTok}[1]{\textcolor[rgb]{0.31,0.60,0.02}{#1}}
\newcommand{\StringTok}[1]{\textcolor[rgb]{0.31,0.60,0.02}{#1}}
\newcommand{\VariableTok}[1]{\textcolor[rgb]{0.00,0.00,0.00}{#1}}
\newcommand{\VerbatimStringTok}[1]{\textcolor[rgb]{0.31,0.60,0.02}{#1}}
\newcommand{\WarningTok}[1]{\textcolor[rgb]{0.56,0.35,0.01}{\textbf{\textit{#1}}}}
\usepackage{graphicx,grffile}
\makeatletter
\def\maxwidth{\ifdim\Gin@nat@width>\linewidth\linewidth\else\Gin@nat@width\fi}
\def\maxheight{\ifdim\Gin@nat@height>\textheight\textheight\else\Gin@nat@height\fi}
\makeatother
% Scale images if necessary, so that they will not overflow the page
% margins by default, and it is still possible to overwrite the defaults
% using explicit options in \includegraphics[width, height, ...]{}
\setkeys{Gin}{width=\maxwidth,height=\maxheight,keepaspectratio}
\IfFileExists{parskip.sty}{%
\usepackage{parskip}
}{% else
\setlength{\parindent}{0pt}
\setlength{\parskip}{6pt plus 2pt minus 1pt}
}
\setlength{\emergencystretch}{3em}  % prevent overfull lines
\providecommand{\tightlist}{%
  \setlength{\itemsep}{0pt}\setlength{\parskip}{0pt}}
\setcounter{secnumdepth}{0}
% Redefines (sub)paragraphs to behave more like sections
\ifx\paragraph\undefined\else
\let\oldparagraph\paragraph
\renewcommand{\paragraph}[1]{\oldparagraph{#1}\mbox{}}
\fi
\ifx\subparagraph\undefined\else
\let\oldsubparagraph\subparagraph
\renewcommand{\subparagraph}[1]{\oldsubparagraph{#1}\mbox{}}
\fi

%%% Use protect on footnotes to avoid problems with footnotes in titles
\let\rmarkdownfootnote\footnote%
\def\footnote{\protect\rmarkdownfootnote}

%%% Change title format to be more compact
\usepackage{titling}

% Create subtitle command for use in maketitle
\newcommand{\subtitle}[1]{
  \posttitle{
    \begin{center}\large#1\end{center}
    }
}

\setlength{\droptitle}{-2em}

  \title{}
    \pretitle{\vspace{\droptitle}}
  \posttitle{}
    \author{}
    \preauthor{}\postauthor{}
    \date{}
    \predate{}\postdate{}
  
\usepackage{booktabs}
\usepackage{longtable}
\usepackage{array}
\usepackage{multirow}
\usepackage[table]{xcolor}
\usepackage{wrapfig}
\usepackage{float}
\usepackage{colortbl}
\usepackage{pdflscape}
\usepackage{tabu}
\usepackage{threeparttable}
\usepackage{threeparttablex}
\usepackage[normalem]{ulem}
\usepackage{makecell}

\begin{document}

\hypertarget{obtencion}{%
\paragraph{Obtención}\label{obtencion}}

Los datos con los que se va a realizar el presente trabajo se obtienen a
través del paquete \texttt{quantmod}. Este paquete se diseñó para
assistir a los \emph{traders} quantitativos en el desarrollo, evaluación
y puesta en funcionamiento de modelos de \emph{trading} basados en la
estadística. (revisar, cita (pckg info)). Concretamente, se utiliza la
función \texttt{getSymbols} para obtener los datos ya que permite cargar
y manejar \texttt{Symbols} en un ambiente especificado (cita (funs
help)). Las posibles fuentes de datos son: Yahoo; Google, aunque
actualmente no funciona; Alphavantage, con la ventaja de que podemos
obtener datos intra-dia; MySQL, FRED, etc. (cita). Para obtener los
datos se procede a utilizar \emph{Yahoo Finance} como la fuente.

\setlength\parskip{5ex}

La metodología para la búsqueda de los distintos stocks con los cuales
entrenar los modelos predictivos empieza con la idea de querer encontrar
empresas representativas dentro de distintos sectores estratégicos y con
amplio impacto en la economía real. Además se pretende que los stocks
utilizados tengan rentabilidad financiera con el objetivo de llevar el
presente trabajo lo más cerca posible de una situación real de
inversión. Para ello se utiliza el portfolio a fecha 01/01/2019 del gran
\emph{gurú} de las finanzas Warren Buffet {[}@warrenbuffet{]}. Dentro de
las numerosas empresas presentes en este portfolio se escogen los 3
\emph{stocks} del NYSE (bolsa de nueva york) y una del NASDAQ.

\begin{Shaded}
\begin{Highlighting}[]
\KeywordTok{library}\NormalTok{(kableExtra)}
\KeywordTok{library}\NormalTok{(dplyr)}
\end{Highlighting}
\end{Shaded}

\begin{verbatim}
## 
## Attaching package: 'dplyr'
\end{verbatim}

\begin{verbatim}
## The following objects are masked from 'package:stats':
## 
##     filter, lag
\end{verbatim}

\begin{verbatim}
## The following objects are masked from 'package:lubridate':
## 
##     intersect, setdiff, union
\end{verbatim}

\begin{verbatim}
## The following objects are masked from 'package:base':
## 
##     intersect, setdiff, setequal, union
\end{verbatim}

\begin{Shaded}
\begin{Highlighting}[]
\KeywordTok{library}\NormalTok{(quantmod)}
\end{Highlighting}
\end{Shaded}

\begin{verbatim}
## Loading required package: xts
\end{verbatim}

\begin{verbatim}
## Loading required package: zoo
\end{verbatim}

\begin{verbatim}
## 
## Attaching package: 'zoo'
\end{verbatim}

\begin{verbatim}
## The following objects are masked from 'package:base':
## 
##     as.Date, as.Date.numeric
\end{verbatim}

\begin{verbatim}
## 
## Attaching package: 'xts'
\end{verbatim}

\begin{verbatim}
## The following objects are masked from 'package:dplyr':
## 
##     first, last
\end{verbatim}

\begin{verbatim}
## Loading required package: TTR
\end{verbatim}

\begin{verbatim}
## Version 0.4-0 included new data defaults. See ?getSymbols.
\end{verbatim}

\begin{Shaded}
\begin{Highlighting}[]
\KeywordTok{kable}\NormalTok{(}\KeywordTok{data.frame}\NormalTok{(}\DataTypeTok{Simbolo=}\KeywordTok{c}\NormalTok{(}\StringTok{"AAPL"}\NormalTok{,}\StringTok{"WFC"}\NormalTok{,}\StringTok{"KO"}\NormalTok{,}\StringTok{"AXP"}\NormalTok{),}
                 \DataTypeTok{Nombre=}\KeywordTok{c}\NormalTok{(}\StringTok{"Apple Inc."}\NormalTok{,}\StringTok{"Wells Fargo & CO"}\NormalTok{,}\StringTok{"Coca-Cola Company"}\NormalTok{,}\StringTok{"American Express CO"}\NormalTok{),}
                 \DataTypeTok{Bolsa=}\KeywordTok{c}\NormalTok{(}\StringTok{"NASDAQ"}\NormalTok{,}\StringTok{"NYSE"}\NormalTok{,}\StringTok{"NYSE"}\NormalTok{,}\StringTok{"NYSE"}\NormalTok{)), }\StringTok{"latex"}\NormalTok{) }\OperatorTok
\KeywordTok{kable_styling}\NormalTok{(}\DataTypeTok{latex_options =} \KeywordTok{c}\NormalTok{(}\StringTok{"basic"}\NormalTok{),}\DataTypeTok{font_size =} \DecValTok{10}\NormalTok{,}\DataTypeTok{bootstrap_options =} \StringTok{"hover"}\NormalTok{)}
\end{Highlighting}
\end{Shaded}

\begin{table}[H]
\centering\begingroup\fontsize{10}{12}\selectfont

\begin{tabular}{l|l|l}
\hline
Simbolo & Nombre & Bolsa\\
\hline
AAPL & Apple Inc. & NASDAQ\\
\hline
WFC & Wells Fargo \& CO & NYSE\\
\hline
KO & Coca-Cola Company & NYSE\\
\hline
AXP & American Express CO & NYSE\\
\hline
\end{tabular}\endgroup{}
\end{table}
\centering
  \captionof{table}{Stocks utilizados}

\setlength\parskip{5ex}
\justifying

Al hecho de que los \emph{stocks} que forman la base de datos formen
parte del portfolio de uno de los \emph{gurús} del mundo de las finanzas
y la inversión, se le suma también el hecho de que las empresas
escogidas son empresas multinacionales con un gran capital que ocupan
una posición destacada sus respectivos mercados, siendo representativas
de cada uno de los sectores en los cuales operan.

Coca - cola (explicar un poco la compañía) etc etc

\hypertarget{descripcion}{%
\paragraph{Descripción}\label{descripcion}}

Después de utilizar la función \texttt{getSymbols} se obtiene una tabla
para cada stock con un formato estandarizado. Se obtienen 5 series
temporales con periodicidad diaria que hacen referencia a los precios de
\textbf{apertura}, \textbf{cierre}, \textbf{máximo}, \textbf{mínimo} y
\textbf{volúmen}. Además se incluye el precio ajustado pero esta
variable no será utilizada en este trabajo. Todas las variables
presentes en la base de datos utilizada son numéricas.

\setlength\parskip{5ex}
\justifying

Los datos utilizados en este trabajo compreden el período 2000 - ?,
siendo el día \texttt{2000-01-01} la primera observación de cada serie
temporal. La última observación de la base de datos es (revisar)

\setlength\parskip{5ex}
\justifying

La función \texttt{str} permite visualizar fácilmente la estructura y
formato que presentan en R las distintas tablas de la base de datos
inicial. En el siguiente output se muestra como ejemplo la empresa
Coca-Cola Company.

\centering

\begin{Shaded}
\begin{Highlighting}[]
\KeywordTok{source}\NormalTok{(}\StringTok{"C:/Users/i0386388/Desktop/tesis/Tesis/load.R"}\NormalTok{)}
\end{Highlighting}
\end{Shaded}

\begin{verbatim}
## 
## Attaching package: 'CombMSC'
\end{verbatim}

\begin{verbatim}
## The following object is masked from 'package:stats':
## 
##     BIC
\end{verbatim}

\begin{verbatim}
## randomForest 4.6-14
\end{verbatim}

\begin{verbatim}
## Type rfNews() to see new features/changes/bug fixes.
\end{verbatim}

\begin{verbatim}
## 
## Attaching package: 'randomForest'
\end{verbatim}

\begin{verbatim}
## The following object is masked from 'package:dplyr':
## 
##     combine
\end{verbatim}

\begin{verbatim}
## The following object is masked from 'package:ggplot2':
## 
##     margin
\end{verbatim}

\begin{verbatim}
## -- Attaching packages ------------------------------------------------------------------------------------------------------------ tidyverse 1.2.1 --
\end{verbatim}

\begin{verbatim}
## v tibble  1.4.2     v purrr   0.2.5
## v tidyr   0.8.1     v stringr 1.3.1
## v readr   1.1.1     v forcats 0.3.0
\end{verbatim}

\begin{verbatim}
## -- Conflicts --------------------------------------------------------------------------------------------------------------- tidyverse_conflicts() --
## x lubridate::as.difftime() masks base::as.difftime()
## x randomForest::combine()  masks dplyr::combine()
## x lubridate::date()        masks base::date()
## x dplyr::filter()          masks stats::filter()
## x xts::first()             masks dplyr::first()
## x lubridate::intersect()   masks base::intersect()
## x dplyr::lag()             masks stats::lag()
## x xts::last()              masks dplyr::last()
## x randomForest::margin()   masks ggplot2::margin()
## x lubridate::setdiff()     masks base::setdiff()
## x lubridate::union()       masks base::union()
\end{verbatim}

\begin{verbatim}
## Loading required package: lattice
\end{verbatim}

\begin{verbatim}
## 
## Attaching package: 'caret'
\end{verbatim}

\begin{verbatim}
## The following object is masked from 'package:purrr':
## 
##     lift
\end{verbatim}

\begin{verbatim}
## 
## Attaching package: 'foreach'
\end{verbatim}

\begin{verbatim}
## The following objects are masked from 'package:purrr':
## 
##     accumulate, when
\end{verbatim}

\begin{verbatim}
## 'getSymbols' currently uses auto.assign=TRUE by default, but will
## use auto.assign=FALSE in 0.5-0. You will still be able to use
## 'loadSymbols' to automatically load data. getOption("getSymbols.env")
## and getOption("getSymbols.auto.assign") will still be checked for
## alternate defaults.
## 
## This message is shown once per session and may be disabled by setting 
## options("getSymbols.warning4.0"=FALSE). See ?getSymbols for details.
\end{verbatim}

\begin{verbatim}
## 
## WARNING: There have been significant changes to Yahoo Finance data.
## Please see the Warning section of '?getSymbols.yahoo' for details.
## 
## This message is shown once per session and may be disabled by setting
## options("getSymbols.yahoo.warning"=FALSE).
\end{verbatim}

\begin{Shaded}
\begin{Highlighting}[]
\NormalTok{prices.KO<-}\KeywordTok{as.data.frame}\NormalTok{(KO)}
\NormalTok{prices.WFC<-}\KeywordTok{as.data.frame}\NormalTok{(WFC)}
\NormalTok{prices.AAPL<-}\KeywordTok{as.data.frame}\NormalTok{(AAPL)}
\NormalTok{prices.AXP<-}\KeywordTok{as.data.frame}\NormalTok{(AXP)}
\KeywordTok{str}\NormalTok{(prices.KO)}
\end{Highlighting}
\end{Shaded}

\begin{verbatim}
## 'data.frame':    4778 obs. of  6 variables:
##  $ KO.Open    : num  29 28.2 28.2 28.5 28.9 ...
##  $ KO.High    : num  29 28.4 28.7 28.8 30.4 ...
##  $ KO.Low     : num  27.6 27.8 28 28.3 28.9 ...
##  $ KO.Close   : num  28.2 28.2 28.5 28.5 30.4 ...
##  $ KO.Volume  : num  10997000 7308000 9457400 7129200 11474000 ...
##  $ KO.Adjusted: num  12.4 12.4 12.5 12.5 13.3 ...
\end{verbatim}

\setlength\parskip{5ex}
\justifying

\emph{Tablas descriptivas}

Se explora descriptivamente los datos analizando los estadísticos
descriptivos . Para cada empresa, se obtienen los distintos estadísticos
de cada una de las variables descritas previamente. Para ello se
elaboran 4 tablas que hacen referencia a los distintos estadísticos,
calculados sobre los distintos precios y el volúmen, para una misma
empresa.

\begin{Shaded}
\begin{Highlighting}[]
\NormalTok{a<-}\KeywordTok{format}\NormalTok{(}\KeywordTok{data.frame}\NormalTok{(}\KeywordTok{paste0}\NormalTok{(}\KeywordTok{round}\NormalTok{(}\KeywordTok{summary}\NormalTok{(prices.KO}\OperatorTok{$}\NormalTok{KO.Open),}\DecValTok{2}\NormalTok{)) }\OperatorTok\StringTok{ }\KeywordTok{rbind}\NormalTok{(}\DataTypeTok{a=}\KeywordTok{paste0}\NormalTok{(}\KeywordTok{round}\NormalTok{(}\KeywordTok{summary}\NormalTok{(prices.KO}\OperatorTok{$}\NormalTok{KO.High),}\DecValTok{2}\NormalTok{))) }\OperatorTok\KeywordTok{rbind}\NormalTok{(}\DataTypeTok{b=}\KeywordTok{paste0}\NormalTok{(}\KeywordTok{round}\NormalTok{(}\KeywordTok{summary}\NormalTok{(prices.KO}\OperatorTok{$}\NormalTok{KO.Low),}\DecValTok{2}\NormalTok{))) }\OperatorTok\StringTok{ }\KeywordTok{rbind}\NormalTok{(}\DataTypeTok{c=}\KeywordTok{paste0}\NormalTok{(}\KeywordTok{round}\NormalTok{(}\KeywordTok{summary}\NormalTok{(prices.KO}\OperatorTok{$}\NormalTok{KO.Close),}\DecValTok{2}\NormalTok{))) }\OperatorTok\StringTok{ }\KeywordTok{rbind}\NormalTok{(}\KeywordTok{paste0}\NormalTok{(}\KeywordTok{round}\NormalTok{(}\KeywordTok{summary}\NormalTok{(prices.KO}\OperatorTok{$}\NormalTok{KO.Volume)}\OperatorTok{/}\DecValTok{1000000}\NormalTok{,}\DecValTok{3}\NormalTok{),}\StringTok{"M"}\NormalTok{))),}\DataTypeTok{scientific=}\OtherTok{TRUE}\NormalTok{)}

\KeywordTok{rownames}\NormalTok{(a)<-}\KeywordTok{c}\NormalTok{(}\StringTok{"Open"}\NormalTok{,}\StringTok{"High"}\NormalTok{,}\StringTok{"Low"}\NormalTok{,}\StringTok{"Close"}\NormalTok{,}\StringTok{"Volume"}\NormalTok{)}
\KeywordTok{names}\NormalTok{(a)<-}\KeywordTok{names}\NormalTok{(}\KeywordTok{summary}\NormalTok{(prices.KO}\OperatorTok{$}\NormalTok{KO.Open))}

\KeywordTok{kable}\NormalTok{(a,}\StringTok{"latex"}\NormalTok{,}\DataTypeTok{digits =} \DecValTok{2}\NormalTok{) }\OperatorTok\StringTok{ }
\StringTok{  }\KeywordTok{kable_styling}\NormalTok{(}\DataTypeTok{font_size =} \DecValTok{10}\NormalTok{,}\DataTypeTok{latex_options =} \KeywordTok{c}\NormalTok{(}\StringTok{"basic"}\NormalTok{))}
\end{Highlighting}
\end{Shaded}

\begin{table}[H]
\centering\begingroup\fontsize{10}{12}\selectfont

\begin{tabular}{l|l|l|l|l|l|l}
\hline
  & Min. & 1st Qu. & Median & Mean & 3rd Qu. & Max.\\
\hline
Open & 18.55 & 23.49 & 28.75 & 31.43 & 40.4 & 50.82\\
\hline
High & 18.8 & 23.72 & 29.01 & 31.66 & 40.6 & 50.84\\
\hline
Low & 18.5 & 23.26 & 28.47 & 31.19 & 40.15 & 50.25\\
\hline
Close & 18.54 & 23.5 & 28.77 & 31.44 & 40.39 & 50.51\\
\hline
Volume & 2.147M & 9.821M & 12.972M & 14.692M & 17.49M & 124.169M\\
\hline
\end{tabular}\endgroup{}
\end{table}
  \captionof{table}{Estadísticos descriptivos para los distintos precios de Coca-Cola Company}

\setlength\parskip{5ex}
\justifying

Como se puede apreciar en la tabla 5.2 el rango que han tomado los
precios de cierre para la empresa Coca-Cola CO. en el período estudiado
se mueve entre 18.54 y 50.51. El precio medio de cierre para el período
estudiado es de 31.44 dólares.

\begin{Shaded}
\begin{Highlighting}[]
\NormalTok{a<-}\KeywordTok{format}\NormalTok{(}\KeywordTok{data.frame}\NormalTok{(}\KeywordTok{paste0}\NormalTok{(}\KeywordTok{round}\NormalTok{(}\KeywordTok{summary}\NormalTok{(prices.AAPL}\OperatorTok{$}\NormalTok{AAPL.Open),}\DecValTok{2}\NormalTok{)) }\OperatorTok\StringTok{ }\KeywordTok{rbind}\NormalTok{(}\DataTypeTok{a=}\KeywordTok{paste0}\NormalTok{(}\KeywordTok{round}\NormalTok{(}\KeywordTok{summary}\NormalTok{(prices.AAPL}\OperatorTok{$}\NormalTok{AAPL.High),}\DecValTok{2}\NormalTok{))) }\OperatorTok\KeywordTok{rbind}\NormalTok{(}\DataTypeTok{b=}\KeywordTok{paste0}\NormalTok{(}\KeywordTok{round}\NormalTok{(}\KeywordTok{summary}\NormalTok{(prices.AAPL}\OperatorTok{$}\NormalTok{AAPL.Low),}\DecValTok{2}\NormalTok{))) }\OperatorTok\StringTok{ }\KeywordTok{rbind}\NormalTok{(}\DataTypeTok{c=}\KeywordTok{paste0}\NormalTok{(}\KeywordTok{round}\NormalTok{(}\KeywordTok{summary}\NormalTok{(prices.AAPL}\OperatorTok{$}\NormalTok{AAPL.Close),}\DecValTok{2}\NormalTok{))) }\OperatorTok\StringTok{ }\KeywordTok{rbind}\NormalTok{(}\KeywordTok{paste0}\NormalTok{(}\KeywordTok{round}\NormalTok{(}\KeywordTok{summary}\NormalTok{(prices.AAPL}\OperatorTok{$}\NormalTok{AAPL.Volume)}\OperatorTok{/}\DecValTok{1000000}\NormalTok{,}\DecValTok{3}\NormalTok{),}\StringTok{"M"}\NormalTok{))),}\DataTypeTok{scientific=}\OtherTok{TRUE}\NormalTok{)}

\KeywordTok{rownames}\NormalTok{(a)<-}\KeywordTok{c}\NormalTok{(}\StringTok{"Open"}\NormalTok{,}\StringTok{"High"}\NormalTok{,}\StringTok{"Low"}\NormalTok{,}\StringTok{"Close"}\NormalTok{,}\StringTok{"Volume"}\NormalTok{)}
\KeywordTok{names}\NormalTok{(a)<-}\KeywordTok{names}\NormalTok{(}\KeywordTok{summary}\NormalTok{(prices.KO}\OperatorTok{$}\NormalTok{KO.Open))}
\KeywordTok{kable}\NormalTok{(a,}\StringTok{"latex"}\NormalTok{) }\OperatorTok\StringTok{ }
\StringTok{  }\KeywordTok{kable_styling}\NormalTok{(}\DataTypeTok{font_size =} \DecValTok{10}\NormalTok{,}\DataTypeTok{latex_options =} \KeywordTok{c}\NormalTok{(}\StringTok{"basic"}\NormalTok{))}
\end{Highlighting}
\end{Shaded}

\begin{table}[H]
\centering\begingroup\fontsize{10}{12}\selectfont

\begin{tabular}{l|l|l|l|l|l|l}
\hline
  & Min. & 1st Qu. & Median & Mean & 3rd Qu. & Max.\\
\hline
Open & 0.93 & 4.37 & 26.13 & 51.65 & 90.52 & 230.78\\
\hline
High & 0.94 & 4.47 & 26.53 & 52.13 & 91.37 & 233.47\\
\hline
Low & 0.91 & 4.21 & 25.66 & 51.12 & 89.72 & 229.78\\
\hline
Close & 0.94 & 4.36 & 26.1 & 51.64 & 90.52 & 232.07\\
\hline
Volume & 9.835M & 52.133M & 93.003M & 119.542M & 156.041M & 1855.41M\\
\hline
\end{tabular}\endgroup{}
\end{table}
  \captionof{table}{Estadísticos descriptivos para los distintos precios de Apple Inc.}

\setlength\parskip{5ex}
\justifying

Como se puede apreciar en la tabla 5.3 el rango que han tomado los
precios de cierre para la empresa Apple Inc.~en el período estudiado se
mueve entre 0.94 y 232.07. El precio medio de cierre para el período
estudiado es de 51.64 dólares, superior al de la empresa Coca-Cola CO.

\begin{Shaded}
\begin{Highlighting}[]
\NormalTok{a<-}\KeywordTok{format}\NormalTok{(}\KeywordTok{data.frame}\NormalTok{(}\KeywordTok{paste0}\NormalTok{(}\KeywordTok{round}\NormalTok{(}\KeywordTok{summary}\NormalTok{(prices.AXP}\OperatorTok{$}\NormalTok{AXP.Open),}\DecValTok{2}\NormalTok{)) }\OperatorTok\StringTok{ }\KeywordTok{rbind}\NormalTok{(}\DataTypeTok{a=}\KeywordTok{paste0}\NormalTok{(}\KeywordTok{round}\NormalTok{(}\KeywordTok{summary}\NormalTok{(prices.AXP}\OperatorTok{$}\NormalTok{AXP.High),}\DecValTok{2}\NormalTok{))) }\OperatorTok\KeywordTok{rbind}\NormalTok{(}\DataTypeTok{b=}\KeywordTok{paste0}\NormalTok{(}\KeywordTok{round}\NormalTok{(}\KeywordTok{summary}\NormalTok{(prices.AXP}\OperatorTok{$}\NormalTok{AXP.Low),}\DecValTok{2}\NormalTok{))) }\OperatorTok\StringTok{ }\KeywordTok{rbind}\NormalTok{(}\DataTypeTok{c=}\KeywordTok{paste0}\NormalTok{(}\KeywordTok{round}\NormalTok{(}\KeywordTok{summary}\NormalTok{(prices.AXP}\OperatorTok{$}\NormalTok{AXP.Close),}\DecValTok{2}\NormalTok{))) }\OperatorTok\StringTok{ }\KeywordTok{rbind}\NormalTok{(}\KeywordTok{paste0}\NormalTok{(}\KeywordTok{round}\NormalTok{(}\KeywordTok{summary}\NormalTok{(prices.AXP}\OperatorTok{$}\NormalTok{AXP.Volume)}\OperatorTok{/}\DecValTok{1000000}\NormalTok{,}\DecValTok{3}\NormalTok{),}\StringTok{"M"}\NormalTok{))),}\DataTypeTok{scientific=}\OtherTok{TRUE}\NormalTok{)}

\KeywordTok{rownames}\NormalTok{(a)<-}\KeywordTok{c}\NormalTok{(}\StringTok{"Open"}\NormalTok{,}\StringTok{"High"}\NormalTok{,}\StringTok{"Low"}\NormalTok{,}\StringTok{"Close"}\NormalTok{,}\StringTok{"Volume"}\NormalTok{)}
\KeywordTok{names}\NormalTok{(a)<-}\KeywordTok{names}\NormalTok{(}\KeywordTok{summary}\NormalTok{(prices.KO}\OperatorTok{$}\NormalTok{KO.Open))}
\KeywordTok{kable}\NormalTok{(a,}\StringTok{"latex"}\NormalTok{,}\DataTypeTok{digits =} \DecValTok{2}\NormalTok{) }\OperatorTok\StringTok{ }
\StringTok{  }\KeywordTok{kable_styling}\NormalTok{(}\DataTypeTok{font_size =} \DecValTok{10}\NormalTok{,}\DataTypeTok{latex_options =} \KeywordTok{c}\NormalTok{(}\StringTok{"basic"}\NormalTok{))}
\end{Highlighting}
\end{Shaded}

\begin{table}[H]
\centering\begingroup\fontsize{10}{12}\selectfont

\begin{tabular}{l|l|l|l|l|l|l}
\hline
  & Min. & 1st Qu. & Median & Mean & 3rd Qu. & Max.\\
\hline
Open & 9.99 & 41.14 & 50.19 & 55.59 & 69.94 & 113.99\\
\hline
High & 10.66 & 41.58 & 50.89 & 56.16 & 70.45 & 114.55\\
\hline
Low & 9.71 & 40.5 & 49.72 & 55 & 69.56 & 112\\
\hline
Close & 10.26 & 41.04 & 50.22 & 55.59 & 70.08 & 112.89\\
\hline
Volume & 0.837M & 3.85M & 5.314M & 6.979M & 7.965M & 90.337M\\
\hline
\end{tabular}\endgroup{}
\end{table}
  \captionof{table}{Estadísticos descriptivos para los distintos precios de American Express CO.}

\setlength\parskip{5ex}
\justifying

Como se puede apreciar en la tabla 5.4 el rango que han tomado los
precios de cierre para la American Express CO. en el período estudiado
se mueve entre 10.26 y 112.89. El precio medio de cierre para el período
estudiado es de 55.59 dólares, ligeramente superior al de la empresa
Apple Inc..

\begin{Shaded}
\begin{Highlighting}[]
\NormalTok{a<-}\KeywordTok{format}\NormalTok{(}\KeywordTok{data.frame}\NormalTok{(}\KeywordTok{paste0}\NormalTok{(}\KeywordTok{round}\NormalTok{(}\KeywordTok{summary}\NormalTok{(prices.WFC}\OperatorTok{$}\NormalTok{WFC.Open),}\DecValTok{2}\NormalTok{)) }\OperatorTok\StringTok{ }\KeywordTok{rbind}\NormalTok{(}\DataTypeTok{a=}\KeywordTok{paste0}\NormalTok{(}\KeywordTok{round}\NormalTok{(}\KeywordTok{summary}\NormalTok{(prices.WFC}\OperatorTok{$}\NormalTok{WFC.High),}\DecValTok{2}\NormalTok{))) }\OperatorTok\KeywordTok{rbind}\NormalTok{(}\DataTypeTok{b=}\KeywordTok{paste0}\NormalTok{(}\KeywordTok{round}\NormalTok{(}\KeywordTok{summary}\NormalTok{(prices.WFC}\OperatorTok{$}\NormalTok{WFC.Low),}\DecValTok{2}\NormalTok{))) }\OperatorTok\StringTok{ }\KeywordTok{rbind}\NormalTok{(}\DataTypeTok{c=}\KeywordTok{paste0}\NormalTok{(}\KeywordTok{round}\NormalTok{(}\KeywordTok{summary}\NormalTok{(prices.WFC}\OperatorTok{$}\NormalTok{WFC.Close),}\DecValTok{2}\NormalTok{))) }\OperatorTok\StringTok{ }\KeywordTok{rbind}\NormalTok{(}\KeywordTok{paste0}\NormalTok{(}\KeywordTok{round}\NormalTok{(}\KeywordTok{summary}\NormalTok{(prices.WFC}\OperatorTok{$}\NormalTok{WFC.Volume)}\OperatorTok{/}\DecValTok{1000000}\NormalTok{,}\DecValTok{3}\NormalTok{),}\StringTok{"M"}\NormalTok{))),}\DataTypeTok{scientific=}\OtherTok{TRUE}\NormalTok{)}

\KeywordTok{rownames}\NormalTok{(a)<-}\KeywordTok{c}\NormalTok{(}\StringTok{"Open"}\NormalTok{,}\StringTok{"High"}\NormalTok{,}\StringTok{"Low"}\NormalTok{,}\StringTok{"Close"}\NormalTok{,}\StringTok{"Volume"}\NormalTok{)}
\KeywordTok{names}\NormalTok{(a)<-}\KeywordTok{names}\NormalTok{(}\KeywordTok{summary}\NormalTok{(prices.KO}\OperatorTok{$}\NormalTok{KO.Open))}
\KeywordTok{kable}\NormalTok{(a,}\StringTok{"latex"}\NormalTok{,}\DataTypeTok{digits =} \DecValTok{2}\NormalTok{) }\OperatorTok\StringTok{ }
\StringTok{  }\KeywordTok{kable_styling}\NormalTok{(}\DataTypeTok{font_size =} \DecValTok{10}\NormalTok{,}\DataTypeTok{latex_options =} \KeywordTok{c}\NormalTok{(}\StringTok{"basic"}\NormalTok{))}
\end{Highlighting}
\end{Shaded}

\begin{table}[H]
\centering\begingroup\fontsize{10}{12}\selectfont

\begin{tabular}{l|l|l|l|l|l|l}
\hline
  & Min. & 1st Qu. & Median & Mean & 3rd Qu. & Max.\\
\hline
Open & 8.65 & 25.95 & 31.19 & 35.19 & 45.82 & 65.89\\
\hline
High & 8.94 & 26.29 & 31.48 & 35.56 & 46.23 & 66.31\\
\hline
Low & 7.8 & 25.54 & 30.84 & 34.82 & 45.47 & 65.66\\
\hline
Close & 8.12 & 25.91 & 31.2 & 35.2 & 45.97 & 65.93\\
\hline
Volume & 1.774M & 9.129M & 15.235M & 23.412M & 27.088M & 478.737M\\
\hline
\end{tabular}\endgroup{}
\end{table}
  \captionof{table}{Estadísticos descriptivos para los distintos precios de Wells Fargo and CO.}

\setlength\parskip{5ex}
\justifying

Como se puede apreciar en la tabla 5.4 el rango que han tomado los
precios de cierre para la empresa Wells Fargo and CO. en el período
estudiado se mueve entre 8.12 y 65.93. El precio medio de cierre para el
período estudiado es de 35.2 dólares, ligeramente superior al de
Coca-Cola CO. inferior al de \texttt{AAPL} y \texttt{AXP}.

\emph{Visualización gráfica}

Seguidamente se explora gráficamente el precio de cierre para cada una
de las empresas, al ser éste la variable a partir de la cual se
calculará la variable respuesta sobre la que hacer predicción. El
objetivo de este tipo de análisis es poder visualizar el tipo de
evolución que han presentado los distintos \emph{stocks} durante el
período de estudio.

\setlength\parskip{5ex}
\justifying

En el siguiente gráfico se representa el precio de cierre de Coca-Cola
Company para el período comprendido entre el 03/01/2000 y 28/12/2018.

\begin{Shaded}
\begin{Highlighting}[]
\KeywordTok{chartSeries}\NormalTok{(KO}\OperatorTok{$}\NormalTok{KO.Close,}\DataTypeTok{name=}\StringTok{"KO Close"}\NormalTok{,}\DataTypeTok{color.vol =}\NormalTok{ T)}
\end{Highlighting}
\end{Shaded}

\begin{center}\includegraphics{descripcion_obtencion_files/figure-latex/unnamed-chunk-8-1} \end{center}
  \captionof{figure}{Precio de cierre de Coca-Cola Company 03/01/2000 - 28/12/2018}

\setlength\parskip{5ex}
\justifying

En el siguiente gráfico se representa el precio de cierre de Apple
Inc.~para el período comprendido entre el 03/01/2000 y 28/12/2018.

\begin{Shaded}
\begin{Highlighting}[]
\KeywordTok{chartSeries}\NormalTok{(AAPL}\OperatorTok{$}\NormalTok{AAPL.Close,}\StringTok{"candlesticks"}\NormalTok{,}\DataTypeTok{name=}\StringTok{"AAPL Close price"}\NormalTok{,}\DataTypeTok{color.vol =}\NormalTok{ T)}
\end{Highlighting}
\end{Shaded}

\begin{center}\includegraphics{descripcion_obtencion_files/figure-latex/unnamed-chunk-9-1} \end{center}
  \captionof{figure}{Precio de cierre de Apple Inc. 03/01/2000 - 28/12/2018}

\setlength\parskip{5ex}
\justifying

En el siguiente gráfico se representa el precio de cierre de Wells Fargo
and CO. para el período comprendido entre el 03/01/2000 y 28/12/2018.

\begin{Shaded}
\begin{Highlighting}[]
\KeywordTok{chartSeries}\NormalTok{(AXP}\OperatorTok{$}\NormalTok{AXP.Close,}\StringTok{"candlesticks"}\NormalTok{,}\DataTypeTok{name=}\StringTok{"AXP Close price"}\NormalTok{,}\DataTypeTok{color.vol =}\NormalTok{ T)}
\end{Highlighting}
\end{Shaded}

\begin{center}\includegraphics{descripcion_obtencion_files/figure-latex/unnamed-chunk-10-1} \end{center}
\centering
  \captionof{figure}{Precio de cierre de American Express CO. 03/01/2000 - 28/12/2018}

\setlength\parskip{5ex}
\justifying

En el siguiente gráfico se representa el precio de cierre de Wells Fargo
and CO. para el período comprendido entre el 03/01/2000 y 28/12/2018.

\begin{Shaded}
\begin{Highlighting}[]
\KeywordTok{chartSeries}\NormalTok{(WFC}\OperatorTok{$}\NormalTok{WFC.Close,}\StringTok{"candlesticks"}\NormalTok{,}\DataTypeTok{name=}\StringTok{"WFC Close price"}\NormalTok{,}\DataTypeTok{color.vol =}\NormalTok{ T)}
\end{Highlighting}
\end{Shaded}

\begin{center}\includegraphics{descripcion_obtencion_files/figure-latex/unnamed-chunk-11-1} \end{center}
\centering
  \captionof{figure}{Precio de cierre de Wells Fargo and CO. 03/01/2000 - 28/12/2018}

\setlength\parskip{5ex}
\justifying

A partir de esta exploración gráfica se descubre que 3 de las 4 empresas
(KO, AXP y WFC) presentan un comportamiento relativamente similar. Como
se puede observar en la figura 5.1, 5.3 y 5.4 la evolución de
\texttt{KO,\ AXP\ y\ WFC} en el período estudiado presenta una
tendencia, en general, creciente. Si se analiza en detalle el lector
puede detectar 3 etapas claramente destacada en la evolución de estos
stocks, siendo la primera de ellas distinta en el caso de Wells Fargo \&
CO. Estos 3 períodos claramente diferenciados son los correspondientes a
los años 2000-2007, 2008-2010 y 2011-2018.

\textbf{2000-2006}

En el caso de las empresas Coca-Cola CO. y American Express CO. la
primera etapa (2000 - inicios de 2006) corresponde a la relajación de
los mercados financieros posterior al \emph{boom de las .com}. La
segunda mitad de los 90 fueron una época de expansión económica en la
cual se produjo un boom financiero en USA {[}ver @stockboomJermann{]}
que se relajo a partir de finales de milenio.

\textbf{2006-2008}

Esta relajación fué seguida por la segunda etapa, un período de
crecimiento económico y de los mercados financieros que se produjo a
partir del 2006 y hasta el 2008. Esta etapa de crecimiento fué previa a
la gran recesión mundial que tuvo lugar a partir del año 2008 {[}ver
@crisisreasons, @crisisreasons2, @crisisreasons3 pp. 77-110 para más
información{]}

\setlength\parskip{5ex}
\justifying

\emph{Descriptiva financiera}

CONTAR LAS VECES QUE EL PRECIO EN EL DÍA T ES MAYOR/INFERIOR A T-1 QUE
EL PRECIO A BAJADO EN EL PRECIO SIN ALISAR. DESPUES CON EL PRECIO
ALISADO Y COMPARANDO LA \% DE DÍA

SABEMOS QUE ESTA VARIABLE SE PUEDE PREDECIR MUY BIEN A.K.A PODEMOS
CLASIFICAR MUY BIEN CON INDICADORES TECNICOS LOS DÍAS QUE INCREMENTA O
DECREMENTA EL PRECIO. ESTO PRUEBA QUE LOS INDICADORES TECNICOS SON
UTILES A LA HORA DE HACER INVERSIONES PORQUE, ANALIZADOS CORRECTAMENTE,
``PERMITEN'' SABER SI EN UN DIA T EL PRECIO HABRÁ SUBIDO O NO RESPECTO A
T-1. ESTA NO ES LA VARIABLE RESPUESTA QUE SE UTILIZA EN ESTE TRABAJO, AL
UTILIZAR INDICADORES TECNICOS CALCULADOS CON EL PROPIO EN PRECIO EN T
PARA PODER SABER SI EL PRECIO EN T HA SUBIDO RESPECTO T-1

Seguidamente se muestran las distintas tablas comparativas de los
distintos precios obtenidos inicialmente. Éstas incluyen la media, la
desviación típica o volatilidad del precio y el coeficiente de
variación, el cual es un estadístico utilizado de manera frecuente para
medir la volatilidad de un stock al ser una variable que no tiene en
cuenta las unidades de medida. Esto significa que permite comparar
distintos stocks en cuanto a volatilidad se refiere sin tener en cuenta
la magnitud que éstos presenten.

Este tipo de visualización permite analizar más detenidamente cada uno
de los precios y poder compararlos fácilmente entre todas las empresas.
Es además una buena manera de analizar a priori la predictibilidad de
los distintos precios al poder comparar los distintos coeficientes de
variación. La idea de evaluar la predictibilidad de una serie temporal
con el coeficiente de variación sugiere que es más fácil de predecir una
serie temporal con un comportamiento más estable, esto es, con menos
unidades de desviación típica por unidad de media {[}@CoVMike{]}.

\begin{Shaded}
\begin{Highlighting}[]
\KeywordTok{kable}\NormalTok{(}\KeywordTok{data.frame}\NormalTok{(}\DataTypeTok{Nombre=}\KeywordTok{c}\NormalTok{(}\StringTok{"Apple Inc."}\NormalTok{,}\StringTok{"Wells Fargo & CO"}\NormalTok{,}\StringTok{"Coca-Cola Company"}\NormalTok{,}\StringTok{"American Express CO"}\NormalTok{),}
                 \DataTypeTok{Media=}\KeywordTok{c}\NormalTok{(}\KeywordTok{mean}\NormalTok{(prices.AAPL}\OperatorTok{$}\NormalTok{AAPL.Open),}\KeywordTok{mean}\NormalTok{(prices.WFC}\OperatorTok{$}\NormalTok{WFC.Open),}\KeywordTok{mean}\NormalTok{(prices.KO}\OperatorTok{$}\NormalTok{KO.Open),}\KeywordTok{mean}\NormalTok{(prices.AXP}\OperatorTok{$}\NormalTok{AXP.Open)),}
                 \DataTypeTok{Desv_std=}\KeywordTok{c}\NormalTok{(}\KeywordTok{sd}\NormalTok{(prices.AAPL}\OperatorTok{$}\NormalTok{AAPL.Open),}\KeywordTok{sd}\NormalTok{(prices.WFC}\OperatorTok{$}\NormalTok{WFC.Open),}\KeywordTok{sd}\NormalTok{(prices.KO}\OperatorTok{$}\NormalTok{KO.Open),}\KeywordTok{sd}\NormalTok{(prices.AXP}\OperatorTok{$}\NormalTok{AXP.Open)),}
                 \DataTypeTok{CoV=}\KeywordTok{c}\NormalTok{((}\KeywordTok{sd}\NormalTok{(prices.AAPL}\OperatorTok{$}\NormalTok{AAPL.Open)}\OperatorTok{/}\KeywordTok{mean}\NormalTok{(prices.AAPL}\OperatorTok{$}\NormalTok{AAPL.Open)),(}\KeywordTok{sd}\NormalTok{(prices.WFC}\OperatorTok{$}\NormalTok{WFC.Open)}\OperatorTok{/}\KeywordTok{mean}\NormalTok{(prices.WFC}\OperatorTok{$}\NormalTok{WFC.Open)),(}\KeywordTok{sd}\NormalTok{(prices.KO}\OperatorTok{$}\NormalTok{KO.Open}\OperatorTok{/}\KeywordTok{mean}\NormalTok{(prices.KO}\OperatorTok{$}\NormalTok{KO.Open))),(}\KeywordTok{sd}\NormalTok{(prices.AXP}\OperatorTok{$}\NormalTok{AXP.Open)}\OperatorTok{/}\KeywordTok{mean}\NormalTok{(prices.AXP}\OperatorTok{$}\NormalTok{AXP.Open)))),}\StringTok{"latex"}\NormalTok{) }\OperatorTok
\StringTok{  }\KeywordTok{kable_styling}\NormalTok{(}\DataTypeTok{font_size =} \DecValTok{10}\NormalTok{,}\DataTypeTok{latex_options =} \KeywordTok{c}\NormalTok{(}\StringTok{"basic"}\NormalTok{))}
\end{Highlighting}
\end{Shaded}

\begin{table}[H]
\centering\begingroup\fontsize{10}{12}\selectfont

\begin{tabular}{l|r|r|r}
\hline
Nombre & Media & Desv\_std & CoV\\
\hline
Apple Inc. & 51.64595 & 55.925416 & 1.0828616\\
\hline
Wells Fargo \& CO & 35.19440 & 11.806917 & 0.3354771\\
\hline
Coca-Cola Company & 31.42690 & 8.652774 & 0.2753302\\
\hline
American Express CO & 55.58608 & 21.112421 & 0.3798149\\
\hline
\end{tabular}\endgroup{}
\end{table}
\centering
  \captionof{table}{Estadísticos descriptivos para el precio de apertura (Open).}

COMENTAR TABLAS??

\begin{Shaded}
\begin{Highlighting}[]
\KeywordTok{kable}\NormalTok{(}\KeywordTok{data.frame}\NormalTok{(}\DataTypeTok{Nombre=}\KeywordTok{c}\NormalTok{(}\StringTok{"Apple Inc."}\NormalTok{,}\StringTok{"Wells Fargo & CO"}\NormalTok{,}\StringTok{"Coca-Cola Company"}\NormalTok{,}\StringTok{"American Express CO"}\NormalTok{),}
                 \DataTypeTok{Media=}\KeywordTok{c}\NormalTok{(}\KeywordTok{mean}\NormalTok{(prices.AAPL}\OperatorTok{$}\NormalTok{AAPL.High),}\KeywordTok{mean}\NormalTok{(prices.WFC}\OperatorTok{$}\NormalTok{WFC.High),}\KeywordTok{mean}\NormalTok{(prices.KO}\OperatorTok{$}\NormalTok{KO.High),}\KeywordTok{mean}\NormalTok{(prices.AXP}\OperatorTok{$}\NormalTok{AXP.High)),}
                 \DataTypeTok{Desv_std=}\KeywordTok{c}\NormalTok{(}\KeywordTok{sd}\NormalTok{(prices.AAPL}\OperatorTok{$}\NormalTok{AAPL.High),}\KeywordTok{sd}\NormalTok{(prices.WFC}\OperatorTok{$}\NormalTok{WFC.High),}\KeywordTok{sd}\NormalTok{(prices.KO}\OperatorTok{$}\NormalTok{KO.High),}\KeywordTok{sd}\NormalTok{(prices.AXP}\OperatorTok{$}\NormalTok{AXP.High)),}
                 \DataTypeTok{CoV=}\KeywordTok{c}\NormalTok{((}\KeywordTok{sd}\NormalTok{(prices.AAPL}\OperatorTok{$}\NormalTok{AAPL.High)}\OperatorTok{/}\KeywordTok{mean}\NormalTok{(prices.AAPL}\OperatorTok{$}\NormalTok{AAPL.High)),(}\KeywordTok{sd}\NormalTok{(prices.WFC}\OperatorTok{$}\NormalTok{WFC.High)}\OperatorTok{/}\KeywordTok{mean}\NormalTok{(prices.WFC}\OperatorTok{$}\NormalTok{WFC.High)),(}\KeywordTok{sd}\NormalTok{(prices.KO}\OperatorTok{$}\NormalTok{KO.High}\OperatorTok{/}\KeywordTok{mean}\NormalTok{(prices.KO}\OperatorTok{$}\NormalTok{KO.High))),(}\KeywordTok{sd}\NormalTok{(prices.AXP}\OperatorTok{$}\NormalTok{AXP.High)}\OperatorTok{/}\KeywordTok{mean}\NormalTok{(prices.AXP}\OperatorTok{$}\NormalTok{AXP.High)))), }\StringTok{"latex"}\NormalTok{) }\OperatorTok
\StringTok{  }\KeywordTok{kable_styling}\NormalTok{(}\DataTypeTok{font_size =} \DecValTok{10}\NormalTok{,}\DataTypeTok{latex_options =} \KeywordTok{c}\NormalTok{(}\StringTok{"basic"}\NormalTok{))}
\end{Highlighting}
\end{Shaded}

\begin{table}[H]
\centering\begingroup\fontsize{10}{12}\selectfont

\begin{tabular}{l|r|r|r}
\hline
Nombre & Media & Desv\_std & CoV\\
\hline
Apple Inc. & 52.13167 & 56.386960 & 1.0816258\\
\hline
Wells Fargo \& CO & 35.56184 & 11.819827 & 0.3323739\\
\hline
Coca-Cola Company & 31.66443 & 8.670834 & 0.2738351\\
\hline
American Express CO & 56.16119 & 21.131090 & 0.3762579\\
\hline
\end{tabular}\endgroup{}
\end{table}
\centering
  \captionof{table}{Estadísticos descriptivos para el precio máximo (High).}

\begin{Shaded}
\begin{Highlighting}[]
\KeywordTok{kable}\NormalTok{(}\KeywordTok{data.frame}\NormalTok{(}\DataTypeTok{Nombre=}\KeywordTok{c}\NormalTok{(}\StringTok{"Apple Inc."}\NormalTok{,}\StringTok{"Wells Fargo & CO"}\NormalTok{,}\StringTok{"Coca-Cola Company"}\NormalTok{,}\StringTok{"American Express CO"}\NormalTok{),}
                 \DataTypeTok{Media=}\KeywordTok{c}\NormalTok{(}\KeywordTok{mean}\NormalTok{(prices.AAPL}\OperatorTok{$}\NormalTok{AAPL.Low),}\KeywordTok{mean}\NormalTok{(prices.WFC}\OperatorTok{$}\NormalTok{WFC.Low),}\KeywordTok{mean}\NormalTok{(prices.KO}\OperatorTok{$}\NormalTok{KO.Low),}\KeywordTok{mean}\NormalTok{(prices.AXP}\OperatorTok{$}\NormalTok{AXP.Low)),}
                 \DataTypeTok{Desv_std=}\KeywordTok{c}\NormalTok{(}\KeywordTok{sd}\NormalTok{(prices.AAPL}\OperatorTok{$}\NormalTok{AAPL.Low),}\KeywordTok{sd}\NormalTok{(prices.WFC}\OperatorTok{$}\NormalTok{WFC.Low),}\KeywordTok{sd}\NormalTok{(prices.KO}\OperatorTok{$}\NormalTok{KO.Low),}\KeywordTok{sd}\NormalTok{(prices.AXP}\OperatorTok{$}\NormalTok{AXP.Low)),}
                 \DataTypeTok{CoV=}\KeywordTok{c}\NormalTok{((}\KeywordTok{sd}\NormalTok{(prices.AAPL}\OperatorTok{$}\NormalTok{AAPL.Low)}\OperatorTok{/}\KeywordTok{mean}\NormalTok{(prices.AAPL}\OperatorTok{$}\NormalTok{AAPL.Low)),(}\KeywordTok{sd}\NormalTok{(prices.WFC}\OperatorTok{$}\NormalTok{WFC.Low)}\OperatorTok{/}\KeywordTok{mean}\NormalTok{(prices.WFC}\OperatorTok{$}\NormalTok{WFC.Low)),(}\KeywordTok{sd}\NormalTok{(prices.KO}\OperatorTok{$}\NormalTok{KO.Low}\OperatorTok{/}\KeywordTok{mean}\NormalTok{(prices.KO}\OperatorTok{$}\NormalTok{KO.Low))),(}\KeywordTok{sd}\NormalTok{(prices.AXP}\OperatorTok{$}\NormalTok{AXP.Low)}\OperatorTok{/}\KeywordTok{mean}\NormalTok{(prices.AXP}\OperatorTok{$}\NormalTok{AXP.Low)))), }\StringTok{"latex"}\NormalTok{) }\OperatorTok
\StringTok{  }\KeywordTok{kable_styling}\NormalTok{(}\DataTypeTok{font_size =} \DecValTok{10}\NormalTok{,}\DataTypeTok{latex_options =} \KeywordTok{c}\NormalTok{(}\StringTok{"basic"}\NormalTok{))}
\end{Highlighting}
\end{Shaded}

\begin{table}[H]
\centering\begingroup\fontsize{10}{12}\selectfont

\begin{tabular}{l|r|r|r}
\hline
Nombre & Media & Desv\_std & CoV\\
\hline
Apple Inc. & 51.12417 & 55.450009 & 1.0846143\\
\hline
Wells Fargo \& CO & 34.82003 & 11.805231 & 0.3390357\\
\hline
Coca-Cola Company & 31.18907 & 8.644319 & 0.2771585\\
\hline
American Express CO & 55.00056 & 21.083165 & 0.3833264\\
\hline
\end{tabular}\endgroup{}
\end{table}
\centering
  \captionof{table}{Estadísticos descriptivos para el precio mínimo (Low).}

\begin{Shaded}
\begin{Highlighting}[]
\KeywordTok{kable}\NormalTok{(}\KeywordTok{data.frame}\NormalTok{(}\DataTypeTok{Nombre=}\KeywordTok{c}\NormalTok{(}\StringTok{"Apple Inc."}\NormalTok{,}\StringTok{"Wells Fargo & CO"}\NormalTok{,}\StringTok{"Coca-Cola Company"}\NormalTok{,}\StringTok{"American Express CO"}\NormalTok{),}
                 \DataTypeTok{Media=}\KeywordTok{c}\NormalTok{(}\KeywordTok{mean}\NormalTok{(prices.AAPL}\OperatorTok{$}\NormalTok{AAPL.Close),}\KeywordTok{mean}\NormalTok{(prices.WFC}\OperatorTok{$}\NormalTok{WFC.Close),}\KeywordTok{mean}\NormalTok{(prices.KO}\OperatorTok{$}\NormalTok{KO.Close),}\KeywordTok{mean}\NormalTok{(prices.AXP}\OperatorTok{$}\NormalTok{AXP.Close)),}
                 \DataTypeTok{Desv_std=}\KeywordTok{c}\NormalTok{(}\KeywordTok{sd}\NormalTok{(prices.AAPL}\OperatorTok{$}\NormalTok{AAPL.Close),}\KeywordTok{sd}\NormalTok{(prices.WFC}\OperatorTok{$}\NormalTok{WFC.Close),}\KeywordTok{sd}\NormalTok{(prices.KO}\OperatorTok{$}\NormalTok{KO.Close),}\KeywordTok{sd}\NormalTok{(prices.AXP}\OperatorTok{$}\NormalTok{AXP.Close)),}
                 \DataTypeTok{CoV=}\KeywordTok{c}\NormalTok{((}\KeywordTok{sd}\NormalTok{(prices.AAPL}\OperatorTok{$}\NormalTok{AAPL.Close)}\OperatorTok{/}\KeywordTok{mean}\NormalTok{(prices.AAPL}\OperatorTok{$}\NormalTok{AAPL.Close)),(}\KeywordTok{sd}\NormalTok{(prices.WFC}\OperatorTok{$}\NormalTok{WFC.Close)}\OperatorTok{/}\KeywordTok{mean}\NormalTok{(prices.WFC}\OperatorTok{$}\NormalTok{WFC.Close)),(}\KeywordTok{sd}\NormalTok{(prices.KO}\OperatorTok{$}\NormalTok{KO.Close}\OperatorTok{/}\KeywordTok{mean}\NormalTok{(prices.KO}\OperatorTok{$}\NormalTok{KO.Close))),(}\KeywordTok{sd}\NormalTok{(prices.AXP}\OperatorTok{$}\NormalTok{AXP.Close)}\OperatorTok{/}\KeywordTok{mean}\NormalTok{(prices.AXP}\OperatorTok{$}\NormalTok{AXP.Close)))), }\StringTok{"latex"}\NormalTok{) }\OperatorTok
\StringTok{  }\KeywordTok{kable_styling}\NormalTok{(}\DataTypeTok{font_size =} \DecValTok{10}\NormalTok{,}\DataTypeTok{latex_options =} \KeywordTok{c}\NormalTok{(}\StringTok{"basic"}\NormalTok{))}
\end{Highlighting}
\end{Shaded}

\begin{table}[H]
\centering\begingroup\fontsize{10}{12}\selectfont

\begin{tabular}{l|r|r|r}
\hline
Nombre & Media & Desv\_std & CoV\\
\hline
Apple Inc. & 51.63773 & 55.925371 & 1.0830331\\
\hline
Wells Fargo \& CO & 35.19502 & 11.807079 & 0.3354758\\
\hline
Coca-Cola Company & 31.43855 & 8.656529 & 0.2753476\\
\hline
American Express CO & 55.58724 & 21.099819 & 0.3795803\\
\hline
\end{tabular}\endgroup{}
\end{table}
\centering
  \captionof{table}{Estadísticos descriptivos para el precio de cierre (Close).}

\justifying


\end{document}
