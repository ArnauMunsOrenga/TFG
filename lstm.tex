\documentclass[]{article}
\usepackage{lmodern}
\usepackage{amssymb,amsmath}
\usepackage{ifxetex,ifluatex}
\usepackage{fixltx2e} % provides \textsubscript
\ifnum 0\ifxetex 1\fi\ifluatex 1\fi=0 % if pdftex
  \usepackage[T1]{fontenc}
  \usepackage[utf8]{inputenc}
\else % if luatex or xelatex
  \ifxetex
    \usepackage{mathspec}
  \else
    \usepackage{fontspec}
  \fi
  \defaultfontfeatures{Ligatures=TeX,Scale=MatchLowercase}
\fi
% use upquote if available, for straight quotes in verbatim environments
\IfFileExists{upquote.sty}{\usepackage{upquote}}{}
% use microtype if available
\IfFileExists{microtype.sty}{%
\usepackage{microtype}
\UseMicrotypeSet[protrusion]{basicmath} % disable protrusion for tt fonts
}{}
\usepackage[margin=1in]{geometry}
\usepackage{hyperref}
\hypersetup{unicode=true,
            pdfborder={0 0 0},
            breaklinks=true}
\urlstyle{same}  % don't use monospace font for urls
\usepackage{color}
\usepackage{fancyvrb}
\newcommand{\VerbBar}{|}
\newcommand{\VERB}{\Verb[commandchars=\\\{\}]}
\DefineVerbatimEnvironment{Highlighting}{Verbatim}{commandchars=\\\{\}}
% Add ',fontsize=\small' for more characters per line
\usepackage{framed}
\definecolor{shadecolor}{RGB}{248,248,248}
\newenvironment{Shaded}{\begin{snugshade}}{\end{snugshade}}
\newcommand{\AlertTok}[1]{\textcolor[rgb]{0.94,0.16,0.16}{#1}}
\newcommand{\AnnotationTok}[1]{\textcolor[rgb]{0.56,0.35,0.01}{\textbf{\textit{#1}}}}
\newcommand{\AttributeTok}[1]{\textcolor[rgb]{0.77,0.63,0.00}{#1}}
\newcommand{\BaseNTok}[1]{\textcolor[rgb]{0.00,0.00,0.81}{#1}}
\newcommand{\BuiltInTok}[1]{#1}
\newcommand{\CharTok}[1]{\textcolor[rgb]{0.31,0.60,0.02}{#1}}
\newcommand{\CommentTok}[1]{\textcolor[rgb]{0.56,0.35,0.01}{\textit{#1}}}
\newcommand{\CommentVarTok}[1]{\textcolor[rgb]{0.56,0.35,0.01}{\textbf{\textit{#1}}}}
\newcommand{\ConstantTok}[1]{\textcolor[rgb]{0.00,0.00,0.00}{#1}}
\newcommand{\ControlFlowTok}[1]{\textcolor[rgb]{0.13,0.29,0.53}{\textbf{#1}}}
\newcommand{\DataTypeTok}[1]{\textcolor[rgb]{0.13,0.29,0.53}{#1}}
\newcommand{\DecValTok}[1]{\textcolor[rgb]{0.00,0.00,0.81}{#1}}
\newcommand{\DocumentationTok}[1]{\textcolor[rgb]{0.56,0.35,0.01}{\textbf{\textit{#1}}}}
\newcommand{\ErrorTok}[1]{\textcolor[rgb]{0.64,0.00,0.00}{\textbf{#1}}}
\newcommand{\ExtensionTok}[1]{#1}
\newcommand{\FloatTok}[1]{\textcolor[rgb]{0.00,0.00,0.81}{#1}}
\newcommand{\FunctionTok}[1]{\textcolor[rgb]{0.00,0.00,0.00}{#1}}
\newcommand{\ImportTok}[1]{#1}
\newcommand{\InformationTok}[1]{\textcolor[rgb]{0.56,0.35,0.01}{\textbf{\textit{#1}}}}
\newcommand{\KeywordTok}[1]{\textcolor[rgb]{0.13,0.29,0.53}{\textbf{#1}}}
\newcommand{\NormalTok}[1]{#1}
\newcommand{\OperatorTok}[1]{\textcolor[rgb]{0.81,0.36,0.00}{\textbf{#1}}}
\newcommand{\OtherTok}[1]{\textcolor[rgb]{0.56,0.35,0.01}{#1}}
\newcommand{\PreprocessorTok}[1]{\textcolor[rgb]{0.56,0.35,0.01}{\textit{#1}}}
\newcommand{\RegionMarkerTok}[1]{#1}
\newcommand{\SpecialCharTok}[1]{\textcolor[rgb]{0.00,0.00,0.00}{#1}}
\newcommand{\SpecialStringTok}[1]{\textcolor[rgb]{0.31,0.60,0.02}{#1}}
\newcommand{\StringTok}[1]{\textcolor[rgb]{0.31,0.60,0.02}{#1}}
\newcommand{\VariableTok}[1]{\textcolor[rgb]{0.00,0.00,0.00}{#1}}
\newcommand{\VerbatimStringTok}[1]{\textcolor[rgb]{0.31,0.60,0.02}{#1}}
\newcommand{\WarningTok}[1]{\textcolor[rgb]{0.56,0.35,0.01}{\textbf{\textit{#1}}}}
\usepackage{graphicx,grffile}
\makeatletter
\def\maxwidth{\ifdim\Gin@nat@width>\linewidth\linewidth\else\Gin@nat@width\fi}
\def\maxheight{\ifdim\Gin@nat@height>\textheight\textheight\else\Gin@nat@height\fi}
\makeatother
% Scale images if necessary, so that they will not overflow the page
% margins by default, and it is still possible to overwrite the defaults
% using explicit options in \includegraphics[width, height, ...]{}
\setkeys{Gin}{width=\maxwidth,height=\maxheight,keepaspectratio}
\IfFileExists{parskip.sty}{%
\usepackage{parskip}
}{% else
\setlength{\parindent}{0pt}
\setlength{\parskip}{6pt plus 2pt minus 1pt}
}
\setlength{\emergencystretch}{3em}  % prevent overfull lines
\providecommand{\tightlist}{%
  \setlength{\itemsep}{0pt}\setlength{\parskip}{0pt}}
\setcounter{secnumdepth}{0}
% Redefines (sub)paragraphs to behave more like sections
\ifx\paragraph\undefined\else
\let\oldparagraph\paragraph
\renewcommand{\paragraph}[1]{\oldparagraph{#1}\mbox{}}
\fi
\ifx\subparagraph\undefined\else
\let\oldsubparagraph\subparagraph
\renewcommand{\subparagraph}[1]{\oldsubparagraph{#1}\mbox{}}
\fi

%%% Use protect on footnotes to avoid problems with footnotes in titles
\let\rmarkdownfootnote\footnote%
\def\footnote{\protect\rmarkdownfootnote}

%%% Change title format to be more compact
\usepackage{titling}

% Create subtitle command for use in maketitle
\providecommand{\subtitle}[1]{
  \posttitle{
    \begin{center}\large#1\end{center}
    }
}

\setlength{\droptitle}{-2em}

  \title{}
    \pretitle{\vspace{\droptitle}}
  \posttitle{}
    \author{}
    \preauthor{}\postauthor{}
    \date{}
    \predate{}\postdate{}
  

\begin{document}

En el presente apartado se procede a construir una red neuronal
recurrente tipo LSTM para predecir los precios de cierre de la empresa
Coca Cola Co.. En hecho de escoger esta empresa está justificado por
distintas razones. En primer lugar el hecho de que solo sea una empresa
y no las 4 utilizadas como ejemplo en el apartado V.1 se debe a que la
capacidad computacional de la que se dispone es limitada. El proceso de
entrenamiento de un modelo de estas características requiere de una
elevada capacidad computacional para poder realizarse en un periodo de
tiempo relativamente razonable. En segundo lugar el hecho de que sea
Coca Cola y no una de las otras 3 empresas responde a lo observado en el
apartado V.1.1. Esta empresa es la que ofrece una opción relativamente
balanceada entre rentabilidad y riesgo. En tercer lugar la tendencia
creciente presente en parte de la serie parece ser relativamente fuerte
y la LSTM pueda probablemente captarlo.

Para emprezar se elabora el plan de entrenamiento del modelo lstm. Con
el procedimiento que se detalla a continuación permite obtener una
evaluación del rendimiento de este tipo de modelos en distintos trozos
de la serie temporal. En este caso se contruyen 6 submuestras que
conforman el plan de entrenamiento, cada una con 996 días en la muestra
de entrenamiento y 83 en la muestra de test. El por qué de esta
configuració se explica posteriormente en la parte de ajuste de los
parámetros. El tercer parámetro que define la partición del plan de
entrenamiento es el que controla la separación entre las ventanas
móviles. En este caso se define en 650 días. Esto significa que la
distancia entre el inicio de las series temporales en cada partición
está separado por 650 días. En definitiva el desarrollo que tiene este
apartado es el siguiente: por simplicidad, se procede a entrenar una
LSTM para cada partición creada, utilizando la misma combinación de
hyperparámetros para este modelo. En este apartado no se elabora una
optimización exhaustiva de los hyperparámetros de este modelo, que es de
hecho donde está la gran complejidad de los algoritmos de machine
learning, sino que se define una combinación de los mismos, que se
podría considerar como referencia para futuras optimizaciones. Esto es
así ya que en el momento de la elaboración de este trabajo no se dispone
de la capacidad computacional adecuada para poder elaborar una tarea de
semejante magnitud, al ser el periodo de tiempo considerado
relativamente elevado (18 años) y al ser el número de combinaciones de
hyper parámetros que hay que probar muy elevada. El hecho de entrenar
disintos modelos en distintos periodos de tiempo ofrece la posibilidad
de analizar el rendimiento de estos modelos con distintas predicciones a
lo largo del tiempo. El hecho de reducir el tamaño de la muestra de
entrenamiento para cada LSTM hace que el proceso de entrenamiento sea
mucho más rápido, con la contrapartida de que los modelos que se crean
no están ofreciendo todo su potencial.

\justifying

En la gráfica que se muestra a continuación se pueden ver representadas
las distintas particiones de la serie temporal de los precios de cierre
de Coca-Cola sobre los cuales se va a entrenar una LSTM-RNN con el fin
de testearla en los 83 días de muestra de prueba, graficados en rojo.

\centering

\begin{Shaded}
\begin{Highlighting}[]
\NormalTok{libraries <-}\StringTok{ }\KeywordTok{c}\NormalTok{(}\StringTok{"lubridate"}\NormalTok{,}\StringTok{"tidyverse"}\NormalTok{,}\StringTok{"tidyr"}\NormalTok{,}\StringTok{"forecast"}\NormalTok{,}\StringTok{"ggplot2"}\NormalTok{,}\StringTok{"seasonal"}\NormalTok{,}\StringTok{"tidyverse"}\NormalTok{,}
               \StringTok{"manipulate"}\NormalTok{,}\StringTok{"compiler"}\NormalTok{,}\StringTok{"scales"}\NormalTok{,}\StringTok{"lmtest"}\NormalTok{,}\StringTok{"MASS"}\NormalTok{,}\StringTok{"glmnet"}\NormalTok{,}\StringTok{"forecTheta"}\NormalTok{,}
               \StringTok{"neuralnet"}\NormalTok{,}\StringTok{"tsDyn"}\NormalTok{,}\StringTok{"changepoint"}\NormalTok{,}\StringTok{"RSNNS"}\NormalTok{,}\StringTok{"xgboost"}\NormalTok{,}\StringTok{"foreach"}\NormalTok{,}\StringTok{"doSNOW"}\NormalTok{,}
               \StringTok{"tcltk"}\NormalTok{,}\StringTok{"DescTools"}\NormalTok{,}\StringTok{"rnn"}\NormalTok{,}\StringTok{"glue"}\NormalTok{,}\StringTok{"forcats"}\NormalTok{,}\StringTok{"timetk"}\NormalTok{,}\StringTok{"tidyquant"}\NormalTok{,}\StringTok{"tibbletime"}\NormalTok{,}
               \StringTok{"cowplot"}\NormalTok{,}\StringTok{"recipes"}\NormalTok{,}\StringTok{"rsample"}\NormalTok{,}\StringTok{"yardstick"}\NormalTok{,}\StringTok{"keras"}\NormalTok{,}\StringTok{"quantmod"}\NormalTok{,}\StringTok{"tidyverse"}
\NormalTok{)}
\CommentTok{# check.libraries <- is.element(libraries, installed.packages()[, 1])==FALSE}
\CommentTok{# libraries.to.install <- libraries[check.libraries]}
\CommentTok{# if (length(libraries.to.install!=0)) \{}
\CommentTok{#   install.packages(libraries.to.install)}
\CommentTok{# \}}

\KeywordTok{lapply}\NormalTok{(libraries, require, }\DataTypeTok{character.only=}\OtherTok{TRUE}\NormalTok{)}
\end{Highlighting}
\end{Shaded}

\begin{verbatim}
## Loading required package: lubridate
\end{verbatim}

\begin{verbatim}
## 
## Attaching package: 'lubridate'
\end{verbatim}

\begin{verbatim}
## The following object is masked from 'package:base':
## 
##     date
\end{verbatim}

\begin{verbatim}
## Loading required package: tidyverse
\end{verbatim}

\begin{verbatim}
## -- Attaching packages ----------------------------------------------------------------------------------------------------------------------------- tidyverse 1.2.1 --
\end{verbatim}

\begin{verbatim}
## v ggplot2 3.1.1       v purrr   0.3.2  
## v tibble  2.1.1       v dplyr   0.8.0.1
## v tidyr   0.8.3       v stringr 1.4.0  
## v readr   1.3.1       v forcats 0.4.0
\end{verbatim}

\begin{verbatim}
## -- Conflicts -------------------------------------------------------------------------------------------------------------------------------- tidyverse_conflicts() --
## x lubridate::as.difftime() masks base::as.difftime()
## x lubridate::date()        masks base::date()
## x dplyr::filter()          masks stats::filter()
## x lubridate::intersect()   masks base::intersect()
## x dplyr::lag()             masks stats::lag()
## x lubridate::setdiff()     masks base::setdiff()
## x lubridate::union()       masks base::union()
\end{verbatim}

\begin{verbatim}
## Loading required package: forecast
\end{verbatim}

\begin{verbatim}
## Loading required package: seasonal
\end{verbatim}

\begin{verbatim}
## 
## Attaching package: 'seasonal'
\end{verbatim}

\begin{verbatim}
## The following object is masked from 'package:tibble':
## 
##     view
\end{verbatim}

\begin{verbatim}
## Loading required package: manipulate
\end{verbatim}

\begin{verbatim}
## Loading required package: compiler
\end{verbatim}

\begin{verbatim}
## Loading required package: scales
\end{verbatim}

\begin{verbatim}
## 
## Attaching package: 'scales'
\end{verbatim}

\begin{verbatim}
## The following object is masked from 'package:purrr':
## 
##     discard
\end{verbatim}

\begin{verbatim}
## The following object is masked from 'package:readr':
## 
##     col_factor
\end{verbatim}

\begin{verbatim}
## Loading required package: lmtest
\end{verbatim}

\begin{verbatim}
## Loading required package: zoo
\end{verbatim}

\begin{verbatim}
## 
## Attaching package: 'zoo'
\end{verbatim}

\begin{verbatim}
## The following objects are masked from 'package:base':
## 
##     as.Date, as.Date.numeric
\end{verbatim}

\begin{verbatim}
## Loading required package: MASS
\end{verbatim}

\begin{verbatim}
## 
## Attaching package: 'MASS'
\end{verbatim}

\begin{verbatim}
## The following object is masked from 'package:dplyr':
## 
##     select
\end{verbatim}

\begin{verbatim}
## Loading required package: glmnet
\end{verbatim}

\begin{verbatim}
## Warning in library(package, lib.loc = lib.loc, character.only = TRUE,
## logical.return = TRUE, : there is no package called 'glmnet'
\end{verbatim}

\begin{verbatim}
## Loading required package: forecTheta
\end{verbatim}

\begin{verbatim}
## Warning in library(package, lib.loc = lib.loc, character.only = TRUE,
## logical.return = TRUE, : there is no package called 'forecTheta'
\end{verbatim}

\begin{verbatim}
## Loading required package: neuralnet
\end{verbatim}

\begin{verbatim}
## Warning in library(package, lib.loc = lib.loc, character.only = TRUE,
## logical.return = TRUE, : there is no package called 'neuralnet'
\end{verbatim}

\begin{verbatim}
## Loading required package: tsDyn
\end{verbatim}

\begin{verbatim}
## Warning in library(package, lib.loc = lib.loc, character.only = TRUE,
## logical.return = TRUE, : there is no package called 'tsDyn'
\end{verbatim}

\begin{verbatim}
## Loading required package: changepoint
\end{verbatim}

\begin{verbatim}
## Successfully loaded changepoint package version 2.2.2
##  NOTE: Predefined penalty values changed in version 2.2.  Previous penalty values with a postfix 1 i.e. SIC1 are now without i.e. SIC and previous penalties without a postfix i.e. SIC are now with a postfix 0 i.e. SIC0. See NEWS and help files for further details.
\end{verbatim}

\begin{verbatim}
## Loading required package: RSNNS
\end{verbatim}

\begin{verbatim}
## Warning in library(package, lib.loc = lib.loc, character.only = TRUE,
## logical.return = TRUE, : there is no package called 'RSNNS'
\end{verbatim}

\begin{verbatim}
## Loading required package: xgboost
\end{verbatim}

\begin{verbatim}
## Warning in library(package, lib.loc = lib.loc, character.only = TRUE,
## logical.return = TRUE, : there is no package called 'xgboost'
\end{verbatim}

\begin{verbatim}
## Loading required package: foreach
\end{verbatim}

\begin{verbatim}
## 
## Attaching package: 'foreach'
\end{verbatim}

\begin{verbatim}
## The following objects are masked from 'package:purrr':
## 
##     accumulate, when
\end{verbatim}

\begin{verbatim}
## Loading required package: doSNOW
\end{verbatim}

\begin{verbatim}
## Loading required package: iterators
\end{verbatim}

\begin{verbatim}
## Loading required package: snow
\end{verbatim}

\begin{verbatim}
## Loading required package: tcltk
\end{verbatim}

\begin{verbatim}
## Loading required package: DescTools
\end{verbatim}

\begin{verbatim}
## 
## Attaching package: 'DescTools'
\end{verbatim}

\begin{verbatim}
## The following object is masked from 'package:forecast':
## 
##     BoxCox
\end{verbatim}

\begin{verbatim}
## Loading required package: rnn
\end{verbatim}

\begin{verbatim}
## Warning in library(package, lib.loc = lib.loc, character.only = TRUE,
## logical.return = TRUE, : there is no package called 'rnn'
\end{verbatim}

\begin{verbatim}
## Loading required package: glue
\end{verbatim}

\begin{verbatim}
## 
## Attaching package: 'glue'
\end{verbatim}

\begin{verbatim}
## The following object is masked from 'package:dplyr':
## 
##     collapse
\end{verbatim}

\begin{verbatim}
## Loading required package: timetk
\end{verbatim}

\begin{verbatim}
## Loading required package: tidyquant
\end{verbatim}

\begin{verbatim}
## Loading required package: PerformanceAnalytics
\end{verbatim}

\begin{verbatim}
## Loading required package: xts
\end{verbatim}

\begin{verbatim}
## 
## Attaching package: 'xts'
\end{verbatim}

\begin{verbatim}
## The following objects are masked from 'package:dplyr':
## 
##     first, last
\end{verbatim}

\begin{verbatim}
## 
## Attaching package: 'PerformanceAnalytics'
\end{verbatim}

\begin{verbatim}
## The following object is masked from 'package:graphics':
## 
##     legend
\end{verbatim}

\begin{verbatim}
## Loading required package: quantmod
\end{verbatim}

\begin{verbatim}
## Loading required package: TTR
\end{verbatim}

\begin{verbatim}
## Version 0.4-0 included new data defaults. See ?getSymbols.
\end{verbatim}

\begin{verbatim}
## Loading required package: tibbletime
\end{verbatim}

\begin{verbatim}
## 
## Attaching package: 'tibbletime'
\end{verbatim}

\begin{verbatim}
## The following object is masked from 'package:stats':
## 
##     filter
\end{verbatim}

\begin{verbatim}
## Loading required package: cowplot
\end{verbatim}

\begin{verbatim}
## 
## Attaching package: 'cowplot'
\end{verbatim}

\begin{verbatim}
## The following object is masked from 'package:ggplot2':
## 
##     ggsave
\end{verbatim}

\begin{verbatim}
## Loading required package: recipes
\end{verbatim}

\begin{verbatim}
## 
## Attaching package: 'recipes'
\end{verbatim}

\begin{verbatim}
## The following object is masked from 'package:stringr':
## 
##     fixed
\end{verbatim}

\begin{verbatim}
## The following object is masked from 'package:stats':
## 
##     step
\end{verbatim}

\begin{verbatim}
## Loading required package: rsample
\end{verbatim}

\begin{verbatim}
## Loading required package: yardstick
\end{verbatim}

\begin{verbatim}
## Warning in library(package, lib.loc = lib.loc, character.only = TRUE,
## logical.return = TRUE, : there is no package called 'yardstick'
\end{verbatim}

\begin{verbatim}
## Loading required package: keras
\end{verbatim}

\begin{verbatim}
## 
## Attaching package: 'keras'
\end{verbatim}

\begin{verbatim}
## The following object is masked from 'package:compiler':
## 
##     compile
\end{verbatim}

\begin{Shaded}
\begin{Highlighting}[]
\KeywordTok{library}\NormalTok{(tidyverse)}
\KeywordTok{library}\NormalTok{(quantmod)}
\KeywordTok{getSymbols}\NormalTok{(}\DataTypeTok{Symbols =} \StringTok{"KO"}\NormalTok{, }\DataTypeTok{from =}\StringTok{"2000-01-01"}\NormalTok{,}\DataTypeTok{to=}\StringTok{"2018-12-31"}\NormalTok{)}
\end{Highlighting}
\end{Shaded}

\begin{verbatim}
## 'getSymbols' currently uses auto.assign=TRUE by default, but will
## use auto.assign=FALSE in 0.5-0. You will still be able to use
## 'loadSymbols' to automatically load data. getOption("getSymbols.env")
## and getOption("getSymbols.auto.assign") will still be checked for
## alternate defaults.
## 
## This message is shown once per session and may be disabled by setting 
## options("getSymbols.warning4.0"=FALSE). See ?getSymbols for details.
\end{verbatim}

\begin{Shaded}
\begin{Highlighting}[]
\CommentTok{#install_keras()}
\end{Highlighting}
\end{Shaded}

\justifying

\begin{Shaded}
\begin{Highlighting}[]
\KeywordTok{load}\NormalTok{(}\StringTok{"C:/Users/i0386388/Desktop/tesis/lstm_prices.RData"}\NormalTok{)}
\NormalTok{sample_predictions_lstm_tbl_prices<-sample_predictions_lstm_tbl}
\KeywordTok{load}\NormalTok{(}\StringTok{"C:/Users/i0386388/Desktop/tesis/lstm_return.RData"}\NormalTok{)}
\NormalTok{sample_predictions_lstm_tbl_return<-sample_predictions_lstm_tbl}
\KeywordTok{rm}\NormalTok{(sample_predictions_lstm_tbl)}
\end{Highlighting}
\end{Shaded}

\begin{Shaded}
\begin{Highlighting}[]
\NormalTok{dates<-}\KeywordTok{rownames}\NormalTok{(}\KeywordTok{as.data.frame}\NormalTok{(KO))}

\NormalTok{Exerci}\FloatTok{.0}\NormalTok{<-}\KeywordTok{data.frame}\NormalTok{(}\DataTypeTok{MSales=}\KeywordTok{as.vector}\NormalTok{(KO}\OperatorTok{$}\NormalTok{KO.Close),}\DataTypeTok{StartDate=}\NormalTok{dates)}\CommentTok{#Close prices}

\KeywordTok{rownames}\NormalTok{(Exerci}\FloatTok{.0}\NormalTok{)<-Exerci}\FloatTok{.0}\OperatorTok{$}\NormalTok{StartDate}
\NormalTok{input<-Exerci}\FloatTok{.0} \OperatorTok\StringTok{ }\NormalTok{dplyr}\OperatorTok{::}\KeywordTok{select}\NormalTok{(MSales)}


\NormalTok{input}\FloatTok{.0}\NormalTok{ <-}\StringTok{ }\NormalTok{input }\OperatorTok
\StringTok{  }\KeywordTok{tk_tbl}\NormalTok{() }\OperatorTok
\StringTok{  }\KeywordTok{mutate}\NormalTok{(}\DataTypeTok{index =} \KeywordTok{as_date}\NormalTok{(index)) }\OperatorTok
\StringTok{  }\KeywordTok{as_tbl_time}\NormalTok{(}\DataTypeTok{index =}\NormalTok{ index) }\OperatorTok\StringTok{ }
\StringTok{  }\KeywordTok{rename}\NormalTok{(}\StringTok{"value"}\NormalTok{=MSales)}


\NormalTok{periods_train <-}\StringTok{ }\DecValTok{996} \CommentTok{#train length in each resample}
\NormalTok{periods_test  <-}\StringTok{ }\DecValTok{83}
\NormalTok{skip_span     <-}\StringTok{ }\DecValTok{650}

\NormalTok{rolling_origin_resamples <-}\StringTok{ }\KeywordTok{rolling_origin}\NormalTok{(}
\NormalTok{  input}\FloatTok{.0}\NormalTok{,}
  \DataTypeTok{initial    =}\NormalTok{ periods_train,}
  \DataTypeTok{assess     =}\NormalTok{ periods_test,}
  \DataTypeTok{cumulative =} \OtherTok{FALSE}\NormalTok{,}
  \DataTypeTok{skip       =}\NormalTok{ skip_span}
\NormalTok{)}

\NormalTok{rolling_origin_resamples}
\end{Highlighting}
\end{Shaded}

\begin{Shaded}
\begin{Highlighting}[]
\CommentTok{# Plotting function for a single split}
\NormalTok{plot_split <-}\StringTok{ }\ControlFlowTok{function}\NormalTok{(split, }\DataTypeTok{expand_y_axis =} \OtherTok{TRUE}\NormalTok{, }\DataTypeTok{alpha =} \DecValTok{1}\NormalTok{, }\DataTypeTok{size =} \DecValTok{1}\NormalTok{, }\DataTypeTok{base_size =} \DecValTok{14}\NormalTok{) \{}
  
  \CommentTok{# Manipulate data}
\NormalTok{  train_tbl <-}\StringTok{ }\KeywordTok{training}\NormalTok{(split) }\OperatorTok
\StringTok{    }\KeywordTok{add_column}\NormalTok{(}\DataTypeTok{key =} \StringTok{"training"}\NormalTok{) }
  
\NormalTok{  test_tbl  <-}\StringTok{ }\KeywordTok{testing}\NormalTok{(split) }\OperatorTok
\StringTok{    }\KeywordTok{add_column}\NormalTok{(}\DataTypeTok{key =} \StringTok{"testing"}\NormalTok{) }
  
\NormalTok{  data_manipulated <-}\StringTok{ }\KeywordTok{bind_rows}\NormalTok{(train_tbl, test_tbl) }\OperatorTok
\StringTok{    }\KeywordTok{as_tbl_time}\NormalTok{(}\DataTypeTok{index =}\NormalTok{ index) }\OperatorTok
\StringTok{    }\KeywordTok{mutate}\NormalTok{(}\DataTypeTok{key =} \KeywordTok{fct_relevel}\NormalTok{(key, }\StringTok{"training"}\NormalTok{, }\StringTok{"testing"}\NormalTok{))}
  
  \CommentTok{# Collect attributes}
\NormalTok{  train_time_summary <-}\StringTok{ }\NormalTok{train_tbl }\OperatorTok
\StringTok{    }\KeywordTok{tk_index}\NormalTok{() }\OperatorTok
\StringTok{    }\KeywordTok{tk_get_timeseries_summary}\NormalTok{()}
  
\NormalTok{  test_time_summary <-}\StringTok{ }\NormalTok{test_tbl }\OperatorTok
\StringTok{    }\KeywordTok{tk_index}\NormalTok{() }\OperatorTok
\StringTok{    }\KeywordTok{tk_get_timeseries_summary}\NormalTok{()}
  
  \CommentTok{# Visualize}
\NormalTok{  g <-}\StringTok{ }\NormalTok{data_manipulated }\OperatorTok
\StringTok{    }\KeywordTok{ggplot}\NormalTok{(}\KeywordTok{aes}\NormalTok{(}\DataTypeTok{x =}\NormalTok{ index, }\DataTypeTok{y =}\NormalTok{ value, }\DataTypeTok{color =}\NormalTok{ key, }\DataTypeTok{group =} \DecValTok{1}\NormalTok{)) }\OperatorTok{+}
\StringTok{    }\KeywordTok{geom_line}\NormalTok{(}\DataTypeTok{size =}\NormalTok{ size, }\DataTypeTok{alpha =}\NormalTok{ alpha) }\OperatorTok{+}
\StringTok{    }\KeywordTok{theme_tq}\NormalTok{(}\DataTypeTok{base_size =}\NormalTok{ base_size) }\OperatorTok{+}
\StringTok{    }\KeywordTok{scale_color_tq}\NormalTok{() }\OperatorTok{+}
\StringTok{    }\KeywordTok{labs}\NormalTok{(}
      \DataTypeTok{title    =} \KeywordTok{glue}\NormalTok{(}\StringTok{"Split: \{split$id\}"}\NormalTok{),}
      \DataTypeTok{subtitle =} \KeywordTok{glue}\NormalTok{(}\StringTok{"\{train_time_summary$start\} to \{test_time_summary$end\}"}\NormalTok{),}
      \DataTypeTok{y =} \StringTok{""}\NormalTok{, }\DataTypeTok{x =} \StringTok{""}
\NormalTok{    ) }\OperatorTok{+}
\StringTok{    }\KeywordTok{theme}\NormalTok{(}\DataTypeTok{legend.position =} \StringTok{"none"}\NormalTok{) }
  
  \ControlFlowTok{if}\NormalTok{ (expand_y_axis) \{}
    
\NormalTok{    input}\FloatTok{.0}\NormalTok{_time_summary <-}\StringTok{ }\NormalTok{input}\FloatTok{.0} \OperatorTok\StringTok{ }
\StringTok{      }\KeywordTok{tk_index}\NormalTok{() }\OperatorTok\StringTok{ }
\StringTok{      }\KeywordTok{tk_get_timeseries_summary}\NormalTok{()}
    
\NormalTok{    g <-}\StringTok{ }\NormalTok{g }\OperatorTok{+}
\StringTok{      }\KeywordTok{scale_x_date}\NormalTok{(}\DataTypeTok{limits =} \KeywordTok{c}\NormalTok{(input}\FloatTok{.0}\NormalTok{_time_summary}\OperatorTok{$}\NormalTok{start, }
\NormalTok{                              input}\FloatTok{.0}\NormalTok{_time_summary}\OperatorTok{$}\NormalTok{end))}
\NormalTok{  \}}
  
  \KeywordTok{return}\NormalTok{(g)}
\NormalTok{\}}


\NormalTok{plot_sampling_plan <-}\StringTok{ }\ControlFlowTok{function}\NormalTok{(sampling_tbl, }\DataTypeTok{expand_y_axis =} \OtherTok{TRUE}\NormalTok{, }
                               \DataTypeTok{ncol =} \DecValTok{3}\NormalTok{, }\DataTypeTok{alpha =} \DecValTok{1}\NormalTok{, }\DataTypeTok{size =} \DecValTok{1}\NormalTok{, }\DataTypeTok{base_size =} \DecValTok{14}\NormalTok{, }
                               \DataTypeTok{title =} \StringTok{"Sampling Plan"}\NormalTok{) \{}
  
  \CommentTok{# Map plot_split() to sampling_tbl}
\NormalTok{  sampling_tbl_with_plots <-}\StringTok{ }\NormalTok{sampling_tbl }\OperatorTok
\StringTok{    }\KeywordTok{mutate}\NormalTok{(}\DataTypeTok{gg_plots =} \KeywordTok{map}\NormalTok{(splits, plot_split, }
                          \DataTypeTok{expand_y_axis =}\NormalTok{ expand_y_axis,}
                          \DataTypeTok{alpha =}\NormalTok{ alpha, }\DataTypeTok{base_size =}\NormalTok{ base_size))}
  
  \CommentTok{# Make plots with cowplot}
\NormalTok{  plot_list <-}\StringTok{ }\NormalTok{sampling_tbl_with_plots}\OperatorTok{$}\NormalTok{gg_plots }
  
\NormalTok{  p_temp <-}\StringTok{ }\NormalTok{plot_list[[}\DecValTok{1}\NormalTok{]] }\OperatorTok{+}\StringTok{ }\KeywordTok{theme}\NormalTok{(}\DataTypeTok{legend.position =} \StringTok{"bottom"}\NormalTok{)}
\NormalTok{  legend <-}\StringTok{ }\KeywordTok{get_legend}\NormalTok{(p_temp)}
  
\NormalTok{  p_body  <-}\StringTok{ }\KeywordTok{plot_grid}\NormalTok{(}\DataTypeTok{plotlist =}\NormalTok{ plot_list, }\DataTypeTok{ncol =}\NormalTok{ ncol)}
  
\NormalTok{  p_title <-}\StringTok{ }\KeywordTok{ggdraw}\NormalTok{() }\OperatorTok{+}\StringTok{ }
\StringTok{    }\KeywordTok{draw_label}\NormalTok{(title, }\DataTypeTok{size =} \DecValTok{18}\NormalTok{, }\DataTypeTok{fontface =} \StringTok{"bold"}\NormalTok{, }\DataTypeTok{colour =} \KeywordTok{palette_light}\NormalTok{()[[}\DecValTok{1}\NormalTok{]])}
  
\NormalTok{  g <-}\StringTok{ }\KeywordTok{plot_grid}\NormalTok{(p_title, p_body, legend, }\DataTypeTok{ncol =} \DecValTok{1}\NormalTok{, }\DataTypeTok{rel_heights =} \KeywordTok{c}\NormalTok{(}\FloatTok{0.05}\NormalTok{, }\DecValTok{1}\NormalTok{, }\FloatTok{0.05}\NormalTok{))}
  
  \KeywordTok{return}\NormalTok{(g)}
  
\NormalTok{\}}

\NormalTok{rolling_origin_resamples }\OperatorTok
\StringTok{  }\KeywordTok{plot_sampling_plan}\NormalTok{(}\DataTypeTok{expand_y_axis =}\NormalTok{ T, }\DataTypeTok{ncol =} \DecValTok{3}\NormalTok{, }\DataTypeTok{alpha =} \DecValTok{1}\NormalTok{, }\DataTypeTok{size =} \DecValTok{1}\NormalTok{, }\DataTypeTok{base_size =} \DecValTok{10}\NormalTok{, }
                     \DataTypeTok{title =} \StringTok{"Ventana móbil de muestras de entrenamiento y prueba"}\NormalTok{)}
\end{Highlighting}
\end{Shaded}

\begin{center}\includegraphics{lstm_files/figure-latex/unnamed-chunk-4-1} \end{center}
\centering
  \captionof{figure}{Plan de entrenamiento de la LSTM para los precios de cierre. Fuente: elaboración propia}

\setlength\parskip{5ex}
\justifying

También se muestra el gráfico ampliado.

\begin{Shaded}
\begin{Highlighting}[]
\NormalTok{rolling_origin_resamples }\OperatorTok
\StringTok{  }\KeywordTok{plot_sampling_plan}\NormalTok{(}\DataTypeTok{expand_y_axis =}\NormalTok{ F, }\DataTypeTok{ncol =} \DecValTok{3}\NormalTok{, }\DataTypeTok{alpha =} \DecValTok{1}\NormalTok{, }\DataTypeTok{size =} \DecValTok{1}\NormalTok{, }\DataTypeTok{base_size =} \DecValTok{10}\NormalTok{, }
                     \DataTypeTok{title =} \StringTok{"Ventana móbil de muestras de entrenamiento y prueba. Ampliado"}\NormalTok{)}
\end{Highlighting}
\end{Shaded}

\begin{center}\includegraphics{lstm_files/figure-latex/unnamed-chunk-5-1} \end{center}
\centering
  \captionof{figure}{Plan de entrenamiento de la LSTM para los precios de cierre. Ampliado. Fuente: elaboración propia}

\setlength\parskip{5ex}
\justifying

Una vez se tienen listas las particiones se procede a la creación de los
modelos LSTM sobre cada una de ellas. Cabe decir que la implementación
que se hace en este trabajo de los modelos LSTM se soporta en el entorno
Keras. En este caso se ha decidido utilizar la implementación del
entorno en keras en R, utilizando el paquete \texttt{keras}, en vez de
hacerlo en Python para mantener la integridad de todo el trabajo en
cuanto al software / lenguaje que se utiliza para escribir y realizar
los apartados de la presente tesis. Esta decisión tiene implicaciones ya
que el paquete \texttt{keras}, aunque trabaja ``por debajo'' con el
entorno de programación TensorFlow, no permite tener un control a tan
bajo nivel de todos los hyper parámetros que se pueden optimizar en un
modelo de estas características.

Como se ha apuntado previamente, en el presente trabajo se elaboran los
modelos LSTM con una cierta combinación de parámetros, debido a la alta
capacidad computacional que requiere el hecho de hacer el proceso de
\emph{rolling origin cross validation} que este tipo de modelos
requieren para optimizar sus hyper parámteros. Los pasos en el proceso
de creación de estos modelos se detallan a continuación:

\begin{enumerate}
\def\labelenumi{\roman{enumi})}
\item
  En primer lugar se separan los datos de cada partición en sendas
  muestras de entrenamiento y prueba, teniendo la primera 996
  observaciones, es decir, días con valor en el precio de cierre de la
  empresa Coca Cola, mientras que la muestra de prueba consta de 83
  días. Esta separación entre muestras de entrenamiento y prueba
  responde al hecho que, a causa de la forma requerida del tensor de
  entrada en el modelo, el número de observaciones en la muestra de
  entrenamiento tiene que ser divisible entre el número de observaciones
  de la muestra de prueba, de manera que el resultado sea un número
  entero.
\item
  En segundo lugar se preprocesan los datos. En este caso se aplica en
  primer lugar la raíz cuadrada a los datos. Este proceso ayuda a
  reducir la varianza y eliminar los outliers. En segundo lugar se
  estandarizan los datos, restándoles la media y dividiéndolos por su
  desviación típica. Este proceso también se conoce como escalar y
  centrar los datos.
\item
  En tercer lugar se transforma la forma de los datos. Éstos necesitan
  ser transformados a forma de tensor, ya que es la forma requerida por
  este tipo de modelos. El concepto de tensor se puede pensar como una
  entidad algebraica que generaliza los conceptos de escalar, vector y
  matriz. Se podría entender como un vector de matrices, en este caso.
  En el presente trabajo la forma del tensor se detalla posteriormente
  en este apartado. Se construyen pues los tensores input y output de
  las muestras de entrenamiento y prueba.
\item
  En cuarto lugar se definen los hyper parámetros y se construye la
  función que se aplicará a cada partición para entrenar la LSTM. En
  cuanto a los parámetros, se presenta una restricción añadida a la
  descrita en el apartado i). El cociente entre el número de
  observaciones de la muestra de entrenamiento y el parámetro llamado
  \emph{batch size}, o tamaño del grupo, tiene que resultar también en
  un número entero. Esta regla también se aplica al número de
  observaciones en la muestra de prueba. El hyper parámetro \emph{batch
  size} controla el número de muestras, u observaciones, sobre las que
  se trabaja antes de actualizar los parámetros internos de la LSTM o
  pesos da la red. De hecho, es el número de observaciones que se
  utilizan para hacer las predicciones y así poder calcular el error que
  permite optimizar los pesos de la red. En el presente trabajo este
  parámetro se fija en 1 de manera que se está utilizando sólo 1
  observación a la vez para actualizar los parámetros internos del
  modelo. Esta configuración a la hora de utilizar el algoritmo de
  optimización del descenso del gradiente (Gradient Descent) para
  optimizar los pesos de la red se llama Stochastic Gradient Descent. Se
  sabe que fijar el \emph{batch size} en un valor pequeño aportan ruido
  y añaden un efecto regularizador que permite generalizar mejor.
  {[}@smallbatch{]} presentan unos resultados en los que confirman que
  utilizando un valor pequeño para el parámetro \emph{batch size} se
  consigue obtener una estabilidad en el entrenamiento y una mejor
  generalización, dado una capacidad computacional, a través de un
  amplio abanico de experimentos.Otro de los parámetros que se fijan
  durante el entrenamiento de las LSTM es el número de épocas. Este
  hyper parámetro define el número de veces que el algoritmo va a
  aprender de todo el conjunto de datos. Es decir, terminar una época
  significa que cada muestra u observación en el conjunto de
  entrenamiento ha tenido la oportunidad de actualizar los parámetros
  internos del modelo o pesos de la red. Las épocas se fijan en 100 para
  limitar el tiempo de entrenamiento. También se fijan a 100 las
  unidades, o redes neuronales, internas de la LSTM. Este parámetro
  también se conoce como neuronas y controla la capacidad que tiene la
  LSTM de aprender. A más neuronas, más capacidad tiene la LSTM de
  aprender. Seguidamente se fija para el ajuste de la LSTM es el llamado
  \emph{time steps}.(revisar, 6 tsteps) Este parámetro corresponde a la
  segunda dimensión del tensor de entrada. En este caso los tensores son
  de la forma \(Input = [996,6,1] , \ output=[83,1]\). Este hyper
  parámetro controla el número de observaciones en el pasado sobre las
  de las que la LSTM aprende. Como se ha detallado en el apartado IV.1
  la LSTM es un tipo de red neuronal que permite aprender de periodos
  alejados en el tiempo, ya que es capaz de determinar la cantidad de
  información del pasado que hay que retener. Por añadidura, se fija el
  parámetro \emph{unit\_forget\_bias} a 1. Este hyper parámetro añade 1
  al seso de la \emph{puerta de olvido} al inicializarse. Además añade
  una referencia a Jozefowicz et. al.~apuntando a que lo recomienda
  {[}@keras{]}. Finalmente cabe decir que la LSTM que se construye es de
  tipo \emph{STATEFULL}, tal y como se describe en el apartado IV.1
  (revisar apartado, poner stateful)
\item
  En quinto lugar se entrena de manera iterativa sobre todas las épocas
  la LSTM.
\item
  Con el modelo entrenado, se calcula la predicción sobre la muestra de
  entrenamiento y se construye un \texttt{data\ frame} con los
  resultados reales y los predichos.
\end{enumerate}

Este proceso se aplica de manera iterativa sobre todas las particiones y
se grafican los resultados obtenidos sobre las muestras de prueba. En
rojo se pueden ver las predicciones, mientras que en negro se dibujan
los valores reales del precio de cierre.

\begin{Shaded}
\begin{Highlighting}[]
\NormalTok{calc_mape <-}\StringTok{ }\ControlFlowTok{function}\NormalTok{(prediction_tbl) \{}
  
\NormalTok{  mape_calculation <-}\StringTok{ }\ControlFlowTok{function}\NormalTok{(data) \{}
\NormalTok{    data }\OperatorTok
\StringTok{      }\KeywordTok{spread}\NormalTok{(}\DataTypeTok{key =}\NormalTok{ key, }\DataTypeTok{value =}\NormalTok{ value) }\OperatorTok
\StringTok{      }\NormalTok{dplyr}\OperatorTok{::}\KeywordTok{select}\NormalTok{(}\OperatorTok{-}\NormalTok{index) }\OperatorTok
\StringTok{      }\KeywordTok{filter}\NormalTok{(}\OperatorTok{!}\KeywordTok{is.na}\NormalTok{(predict)) }\OperatorTok
\StringTok{      }\KeywordTok{rename}\NormalTok{(}
        \DataTypeTok{truth    =}\NormalTok{ actual,}
        \DataTypeTok{estimate =}\NormalTok{ predict}
\NormalTok{      ) }\OperatorTok
\StringTok{      }\KeywordTok{filter}\NormalTok{(truth}\OperatorTok{!=}\DecValTok{0}\NormalTok{) }\OperatorTok\StringTok{ }
\StringTok{      }\KeywordTok{mutate}\NormalTok{(}\DataTypeTok{diff=}\KeywordTok{abs}\NormalTok{(truth}\OperatorTok{-}\NormalTok{estimate)) }\OperatorTok
\StringTok{      }\KeywordTok{mutate}\NormalTok{(}\DataTypeTok{coc=}\NormalTok{diff}\OperatorTok{/}\NormalTok{truth) }\OperatorTok\StringTok{ }
\StringTok{      }\KeywordTok{summarise}\NormalTok{(}\DataTypeTok{mape=}\KeywordTok{mean}\NormalTok{(coc))}\OperatorTok{*}\DecValTok{100}
\NormalTok{  \}}
  
\NormalTok{  safe_mape <-}\StringTok{ }\KeywordTok{possibly}\NormalTok{(mape_calculation, }\DataTypeTok{otherwise =} \OtherTok{NA}\NormalTok{)}
  
  \KeywordTok{safe_mape}\NormalTok{(prediction_tbl)}
  
\NormalTok{\}}

\NormalTok{calc_mape_sfi <-}\StringTok{ }\ControlFlowTok{function}\NormalTok{(prediction_tbl) \{}

\NormalTok{    mape_calculation_sfi <-}\StringTok{ }\ControlFlowTok{function}\NormalTok{(data) \{}
\NormalTok{    data }\OperatorTok
\StringTok{      }\KeywordTok{spread}\NormalTok{(}\DataTypeTok{key =}\NormalTok{ key, }\DataTypeTok{value =}\NormalTok{ value) }\OperatorTok
\StringTok{      }\NormalTok{dplyr}\OperatorTok{::}\KeywordTok{select}\NormalTok{(}\OperatorTok{-}\NormalTok{index) }\OperatorTok
\StringTok{      }\KeywordTok{filter}\NormalTok{(}\OperatorTok{!}\KeywordTok{is.na}\NormalTok{(predict)) }\OperatorTok
\StringTok{      }\KeywordTok{rename}\NormalTok{(}
        \DataTypeTok{truth    =}\NormalTok{ actual,}
        \DataTypeTok{estimate =}\NormalTok{ predict}
\NormalTok{      ) }\OperatorTok
\StringTok{      }\KeywordTok{filter}\NormalTok{(truth}\OperatorTok{!=}\DecValTok{0}\NormalTok{) }\OperatorTok\StringTok{ }
\StringTok{      }\KeywordTok{summarise}\NormalTok{(}\DataTypeTok{diff=}\KeywordTok{sum}\NormalTok{(}\KeywordTok{abs}\NormalTok{(truth}\OperatorTok{-}\NormalTok{estimate)),}
                \DataTypeTok{total=}\KeywordTok{sum}\NormalTok{(truth)) }\OperatorTok
\StringTok{      }\KeywordTok{mutate}\NormalTok{(}\DataTypeTok{mape=}\NormalTok{(diff}\OperatorTok{/}\NormalTok{total)}\OperatorTok{*}\DecValTok{100}\NormalTok{) }
\NormalTok{  \}}
  
\NormalTok{  safe_mape1 <-}\StringTok{ }\KeywordTok{possibly}\NormalTok{(mape_calculation_sfi, }\DataTypeTok{otherwise =} \OtherTok{NA}\NormalTok{)}
  
  \KeywordTok{safe_mape1}\NormalTok{(prediction_tbl)}
\NormalTok{\}}




\CommentTok{# Setup single plot function}

\NormalTok{plot_prediction <-}\StringTok{ }\ControlFlowTok{function}\NormalTok{(data, id, }\DataTypeTok{alpha =} \DecValTok{1}\NormalTok{, }\DataTypeTok{size =} \DecValTok{2}\NormalTok{, }\DataTypeTok{base_size =} \DecValTok{6}\NormalTok{) \{}
  
  \CommentTok{#mape_val <- calc_mape(data)}
  
\NormalTok{  g <-}\StringTok{ }\NormalTok{data }\OperatorTok
\StringTok{    }\KeywordTok{ggplot}\NormalTok{(}\KeywordTok{aes}\NormalTok{(index, value, }\DataTypeTok{color =}\NormalTok{ key)) }\OperatorTok{+}
\StringTok{    }\KeywordTok{geom_point}\NormalTok{(}\DataTypeTok{alpha =}\NormalTok{ alpha, }\DataTypeTok{size =}\NormalTok{ size) }\OperatorTok{+}\StringTok{ }
\StringTok{    }\CommentTok{#geom_line(size = size) + }
\StringTok{    }\NormalTok{tidyquant}\OperatorTok{::}\KeywordTok{theme_tq}\NormalTok{(}\DataTypeTok{base_size =}\NormalTok{ base_size) }\OperatorTok{+}
\StringTok{    }\NormalTok{tidyquant}\OperatorTok{::}\KeywordTok{scale_color_tq}\NormalTok{() }\OperatorTok{+}
\StringTok{    }\KeywordTok{theme}\NormalTok{(}\DataTypeTok{legend.position =} \StringTok{"none"}\NormalTok{) }\OperatorTok{+}
\StringTok{    }\KeywordTok{labs}\NormalTok{(}
      \CommentTok{#title = glue("\{id\}, mape: \{round(mape_val, digits = 1)\}"),}
      \DataTypeTok{x =} \StringTok{""}\NormalTok{, }\DataTypeTok{y =} \StringTok{""}
\NormalTok{    )}
  
  \KeywordTok{return}\NormalTok{(g)}
\NormalTok{\}}


\NormalTok{plot_predictions <-}\StringTok{ }\ControlFlowTok{function}\NormalTok{(sampling_tbl, predictions_col, }
                             \DataTypeTok{ncol =} \DecValTok{3}\NormalTok{, }\DataTypeTok{alpha =} \DecValTok{1}\NormalTok{, }\DataTypeTok{size =} \DecValTok{2}\NormalTok{, }\DataTypeTok{base_size =} \DecValTok{18}\NormalTok{,}
                             \DataTypeTok{title =} \StringTok{"Backtested Predictions"}\NormalTok{) \{}
  
\NormalTok{  predictions_col_expr <-}\StringTok{ }\KeywordTok{enquo}\NormalTok{(predictions_col)}
  
  \CommentTok{# Map plot_split() to sampling_tbl}
\NormalTok{  sampling_tbl_with_plots <-}\StringTok{ }\NormalTok{sampling_tbl }\OperatorTok
\StringTok{    }\KeywordTok{mutate}\NormalTok{(}\DataTypeTok{gg_plots =} \KeywordTok{map2}\NormalTok{(}\OperatorTok{!!}\StringTok{ }\NormalTok{predictions_col_expr, id, }
                           \DataTypeTok{.f        =}\NormalTok{ plot_prediction, }
                           \DataTypeTok{alpha     =}\NormalTok{ alpha, }
                           \DataTypeTok{size      =}\NormalTok{ size, }
                           \DataTypeTok{base_size =}\NormalTok{ base_size)) }
  
  \CommentTok{# Make plots with cowplot}
\NormalTok{  plot_list <-}\StringTok{ }\NormalTok{sampling_tbl_with_plots}\OperatorTok{$}\NormalTok{gg_plots }
  
\NormalTok{  p_temp <-}\StringTok{ }\NormalTok{plot_list[[}\DecValTok{1}\NormalTok{]] }\OperatorTok{+}\StringTok{ }\KeywordTok{theme}\NormalTok{(}\DataTypeTok{legend.position =} \StringTok{"bottom"}\NormalTok{)}
\NormalTok{  legend <-}\StringTok{ }\KeywordTok{get_legend}\NormalTok{(p_temp)}
  
\NormalTok{  p_body  <-}\StringTok{ }\KeywordTok{plot_grid}\NormalTok{(}\DataTypeTok{plotlist =}\NormalTok{ plot_list, }\DataTypeTok{ncol =}\NormalTok{ ncol)}
  
  
  
\NormalTok{  p_title <-}\StringTok{ }\KeywordTok{ggdraw}\NormalTok{() }\OperatorTok{+}\StringTok{ }
\StringTok{    }\KeywordTok{draw_label}\NormalTok{(title, }\DataTypeTok{size =} \DecValTok{18}\NormalTok{, }\DataTypeTok{fontface =} \StringTok{"bold"}\NormalTok{, }\DataTypeTok{colour =} \KeywordTok{palette_light}\NormalTok{()[[}\DecValTok{1}\NormalTok{]])}
  
\NormalTok{  g <-}\StringTok{ }\KeywordTok{plot_grid}\NormalTok{(p_title, p_body, legend, }\DataTypeTok{ncol =} \DecValTok{1}\NormalTok{, }\DataTypeTok{rel_heights =} \KeywordTok{c}\NormalTok{(}\FloatTok{0.05}\NormalTok{, }\DecValTok{1}\NormalTok{, }\FloatTok{0.05}\NormalTok{))}
  
  \KeywordTok{return}\NormalTok{(g)}
  
\NormalTok{\}}


\NormalTok{sample_predictions_lstm_tbl_prices }\OperatorTok
\StringTok{  }\KeywordTok{filter}\NormalTok{(id}\OperatorTok\KeywordTok{c}\NormalTok{(}\StringTok{"Slice1"}\NormalTok{, }\StringTok{"Slice2"}\NormalTok{,}\StringTok{"Slice3"}\NormalTok{)) }\OperatorTok\StringTok{ }
\StringTok{  }\KeywordTok{plot_predictions}\NormalTok{(}\DataTypeTok{predictions_col =}\NormalTok{ predict, }\DataTypeTok{alpha =} \FloatTok{0.5}\NormalTok{, }\DataTypeTok{size =} \DecValTok{1}\NormalTok{, }\DataTypeTok{base_size =} \DecValTok{10}\NormalTok{,}
                   \DataTypeTok{title =} \StringTok{"Predicción sobre las muestras de prueba. Particiones 1 a 3"}\NormalTok{)}
\end{Highlighting}
\end{Shaded}

\begin{center}\includegraphics{lstm_files/figure-latex/unnamed-chunk-6-1} \end{center}
\centering
  \captionof{figure}{Resultados sobre muestra de prueba de la LSTM sobre los precios de cierre en particiones 1 a 3. Fuente: elaboración propia}

\setlength\parskip{5ex}
\justifying

\begin{Shaded}
\begin{Highlighting}[]
\NormalTok{sample_predictions_lstm_tbl_prices }\OperatorTok
\StringTok{  }\KeywordTok{filter}\NormalTok{(id}\OperatorTok\KeywordTok{c}\NormalTok{(}\StringTok{"Slice4"}\NormalTok{, }\StringTok{"Slice5"}\NormalTok{,}\StringTok{"Slice6"}\NormalTok{)) }\OperatorTok\StringTok{ }
\StringTok{  }\KeywordTok{plot_predictions}\NormalTok{(}\DataTypeTok{predictions_col =}\NormalTok{ predict, }\DataTypeTok{alpha =} \FloatTok{0.5}\NormalTok{, }\DataTypeTok{size =} \DecValTok{1}\NormalTok{, }\DataTypeTok{base_size =} \DecValTok{10}\NormalTok{,}
                   \DataTypeTok{title =} \StringTok{"Predicción sobre las muestras de prueba. Particiones 4 a 6"}\NormalTok{)}
\end{Highlighting}
\end{Shaded}

\begin{center}\includegraphics{lstm_files/figure-latex/unnamed-chunk-7-1} \end{center}
\centering
  \captionof{figure}{Resultados sobre muestra de prueba de la LSTM sobre los precios de cierre en particiones 4 a 6. Fuente: elaboración propia}

\setlength\parskip{5ex}
\justifying

Seguidamente se presenta la tabla que muestra el cálculo de las
distintas métricas de rendimiento para las distintas particiones:

\begin{Shaded}
\begin{Highlighting}[]
\NormalTok{sample_mape_tbl <-}\StringTok{ }\NormalTok{sample_predictions_lstm_tbl_prices }\OperatorTok
\StringTok{  }\KeywordTok{mutate}\NormalTok{(}\DataTypeTok{mape =} \KeywordTok{map}\NormalTok{(predict, calc_mape)) }\OperatorTok\StringTok{ }
\StringTok{  }\NormalTok{dplyr}\OperatorTok{::}\KeywordTok{select}\NormalTok{(id, mape)}

\CommentTok{#sample_predictions_lstm_tbl_prices$predict[[1]]}
\ControlFlowTok{for}\NormalTok{(i }\ControlFlowTok{in} \DecValTok{1}\OperatorTok{:}\KeywordTok{length}\NormalTok{(sample_predictions_lstm_tbl_prices}\OperatorTok{$}\NormalTok{predict))\{}
\NormalTok{  temp<-}\KeywordTok{calc_mape_sfi}\NormalTok{(sample_predictions_lstm_tbl_prices}\OperatorTok{$}\NormalTok{predict[[i]])}
  \ControlFlowTok{if}\NormalTok{(}\KeywordTok{exists}\NormalTok{(}\StringTok{"results"}\NormalTok{))\{}
\NormalTok{    results<-results }\OperatorTok\StringTok{ }\KeywordTok{rbind}\NormalTok{(temp)}
\NormalTok{  \}}\ControlFlowTok{else}\NormalTok{\{results<-temp\}}
\NormalTok{\}}
\end{Highlighting}
\end{Shaded}

Como se puede observar en base al cálculo de los dos tipos de MAPE, los
resultados parecen bastante buenos. En la peor de las particiones el
modelo es capaz de generar un error percentual absoluto medio de
alrededor del 7\%

\begin{Shaded}
\begin{Highlighting}[]
\CommentTok{# }
\CommentTok{# do.call(rbind.data.frame, sample_mape_tbl$mape) %>% mutate(Split=paste("split",seq(1,6,1))) %>% }
\CommentTok{#   left_join(}
\CommentTok{#     results %>% mutate(key=paste("split",seq(1,6,1))),}
\CommentTok{#     by=c("Split"="key")}
\CommentTok{#   ) %>% }
\CommentTok{#   dplyr::rename("MAPE"=mape.x,"MAPE2"=mape.y) %>% }
\CommentTok{#   dplyr::select(Split,MAPE,MAPE2) %>% }
\CommentTok{#   gather(metrica,valor,2:3) %>% }
\CommentTok{#   ggplot(aes(x=Split,y=valor,fill=metrica),color="black")+}
\CommentTok{#   geom_bar(stat = "identity",position=position_dodge())+}
\CommentTok{#   theme_tq()+}
\CommentTok{#   scale_fill_brewer(palette="Set1")+}
\CommentTok{#   theme(legend.text  = element_text(size=15),}
\CommentTok{#         legend.title = element_text(size=15))+}
\CommentTok{#   ggtitle("Métricas sobre las distintas particiones obtenidas con la LSTM")}
\CommentTok{# }
\CommentTok{# }
\CommentTok{# sample_predictions_lstm_tbl_prices$predict[[1]]->data}
\CommentTok{# data %>%}
\CommentTok{#     spread(key = key, value = value) %>%}
\CommentTok{#     dplyr::select(-index) %>%}
\CommentTok{#     filter(!is.na(predict)) %>%}
\CommentTok{#     rename(}
\CommentTok{#         truth    = actual,}
\CommentTok{#         estimate = predict}
\CommentTok{#     ) %>%}
\CommentTok{#     filter(truth!=0)->a}
\CommentTok{# }
\CommentTok{# MAPE(x = a$estimate,ref=a$truth)*100}
\CommentTok{# MAE(x = a$estimate,ref=a$truth)*100}
\end{Highlighting}
\end{Shaded}

\begin{Shaded}
\begin{Highlighting}[]
\NormalTok{sample_predictions_lstm_tbl_return }\OperatorTok
\StringTok{  }\KeywordTok{plot_predictions}\NormalTok{(}\DataTypeTok{predictions_col =}\NormalTok{ predict, }\DataTypeTok{alpha =} \FloatTok{0.5}\NormalTok{, }\DataTypeTok{size =} \DecValTok{1}\NormalTok{, }\DataTypeTok{base_size =} \DecValTok{10}\NormalTok{,}
                   \DataTypeTok{title =} \StringTok{"Keras Stateful LSTM: Prediction"}\NormalTok{)}
\end{Highlighting}
\end{Shaded}

\begin{center}\includegraphics{lstm_files/figure-latex/unnamed-chunk-10-1} \end{center}


\end{document}
