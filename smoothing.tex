\documentclass[]{article}
\usepackage{lmodern}
\usepackage{amssymb,amsmath}
\usepackage{ifxetex,ifluatex}
\usepackage{fixltx2e} % provides \textsubscript
\ifnum 0\ifxetex 1\fi\ifluatex 1\fi=0 % if pdftex
  \usepackage[T1]{fontenc}
  \usepackage[utf8]{inputenc}
\else % if luatex or xelatex
  \ifxetex
    \usepackage{mathspec}
  \else
    \usepackage{fontspec}
  \fi
  \defaultfontfeatures{Ligatures=TeX,Scale=MatchLowercase}
\fi
% use upquote if available, for straight quotes in verbatim environments
\IfFileExists{upquote.sty}{\usepackage{upquote}}{}
% use microtype if available
\IfFileExists{microtype.sty}{%
\usepackage{microtype}
\UseMicrotypeSet[protrusion]{basicmath} % disable protrusion for tt fonts
}{}
\usepackage[margin=1in]{geometry}
\usepackage{hyperref}
\hypersetup{unicode=true,
            pdfborder={0 0 0},
            breaklinks=true}
\urlstyle{same}  % don't use monospace font for urls
\usepackage{color}
\usepackage{fancyvrb}
\newcommand{\VerbBar}{|}
\newcommand{\VERB}{\Verb[commandchars=\\\{\}]}
\DefineVerbatimEnvironment{Highlighting}{Verbatim}{commandchars=\\\{\}}
% Add ',fontsize=\small' for more characters per line
\usepackage{framed}
\definecolor{shadecolor}{RGB}{248,248,248}
\newenvironment{Shaded}{\begin{snugshade}}{\end{snugshade}}
\newcommand{\KeywordTok}[1]{\textcolor[rgb]{0.13,0.29,0.53}{\textbf{#1}}}
\newcommand{\DataTypeTok}[1]{\textcolor[rgb]{0.13,0.29,0.53}{#1}}
\newcommand{\DecValTok}[1]{\textcolor[rgb]{0.00,0.00,0.81}{#1}}
\newcommand{\BaseNTok}[1]{\textcolor[rgb]{0.00,0.00,0.81}{#1}}
\newcommand{\FloatTok}[1]{\textcolor[rgb]{0.00,0.00,0.81}{#1}}
\newcommand{\ConstantTok}[1]{\textcolor[rgb]{0.00,0.00,0.00}{#1}}
\newcommand{\CharTok}[1]{\textcolor[rgb]{0.31,0.60,0.02}{#1}}
\newcommand{\SpecialCharTok}[1]{\textcolor[rgb]{0.00,0.00,0.00}{#1}}
\newcommand{\StringTok}[1]{\textcolor[rgb]{0.31,0.60,0.02}{#1}}
\newcommand{\VerbatimStringTok}[1]{\textcolor[rgb]{0.31,0.60,0.02}{#1}}
\newcommand{\SpecialStringTok}[1]{\textcolor[rgb]{0.31,0.60,0.02}{#1}}
\newcommand{\ImportTok}[1]{#1}
\newcommand{\CommentTok}[1]{\textcolor[rgb]{0.56,0.35,0.01}{\textit{#1}}}
\newcommand{\DocumentationTok}[1]{\textcolor[rgb]{0.56,0.35,0.01}{\textbf{\textit{#1}}}}
\newcommand{\AnnotationTok}[1]{\textcolor[rgb]{0.56,0.35,0.01}{\textbf{\textit{#1}}}}
\newcommand{\CommentVarTok}[1]{\textcolor[rgb]{0.56,0.35,0.01}{\textbf{\textit{#1}}}}
\newcommand{\OtherTok}[1]{\textcolor[rgb]{0.56,0.35,0.01}{#1}}
\newcommand{\FunctionTok}[1]{\textcolor[rgb]{0.00,0.00,0.00}{#1}}
\newcommand{\VariableTok}[1]{\textcolor[rgb]{0.00,0.00,0.00}{#1}}
\newcommand{\ControlFlowTok}[1]{\textcolor[rgb]{0.13,0.29,0.53}{\textbf{#1}}}
\newcommand{\OperatorTok}[1]{\textcolor[rgb]{0.81,0.36,0.00}{\textbf{#1}}}
\newcommand{\BuiltInTok}[1]{#1}
\newcommand{\ExtensionTok}[1]{#1}
\newcommand{\PreprocessorTok}[1]{\textcolor[rgb]{0.56,0.35,0.01}{\textit{#1}}}
\newcommand{\AttributeTok}[1]{\textcolor[rgb]{0.77,0.63,0.00}{#1}}
\newcommand{\RegionMarkerTok}[1]{#1}
\newcommand{\InformationTok}[1]{\textcolor[rgb]{0.56,0.35,0.01}{\textbf{\textit{#1}}}}
\newcommand{\WarningTok}[1]{\textcolor[rgb]{0.56,0.35,0.01}{\textbf{\textit{#1}}}}
\newcommand{\AlertTok}[1]{\textcolor[rgb]{0.94,0.16,0.16}{#1}}
\newcommand{\ErrorTok}[1]{\textcolor[rgb]{0.64,0.00,0.00}{\textbf{#1}}}
\newcommand{\NormalTok}[1]{#1}
\usepackage{graphicx,grffile}
\makeatletter
\def\maxwidth{\ifdim\Gin@nat@width>\linewidth\linewidth\else\Gin@nat@width\fi}
\def\maxheight{\ifdim\Gin@nat@height>\textheight\textheight\else\Gin@nat@height\fi}
\makeatother
% Scale images if necessary, so that they will not overflow the page
% margins by default, and it is still possible to overwrite the defaults
% using explicit options in \includegraphics[width, height, ...]{}
\setkeys{Gin}{width=\maxwidth,height=\maxheight,keepaspectratio}
\IfFileExists{parskip.sty}{%
\usepackage{parskip}
}{% else
\setlength{\parindent}{0pt}
\setlength{\parskip}{6pt plus 2pt minus 1pt}
}
\setlength{\emergencystretch}{3em}  % prevent overfull lines
\providecommand{\tightlist}{%
  \setlength{\itemsep}{0pt}\setlength{\parskip}{0pt}}
\setcounter{secnumdepth}{0}
% Redefines (sub)paragraphs to behave more like sections
\ifx\paragraph\undefined\else
\let\oldparagraph\paragraph
\renewcommand{\paragraph}[1]{\oldparagraph{#1}\mbox{}}
\fi
\ifx\subparagraph\undefined\else
\let\oldsubparagraph\subparagraph
\renewcommand{\subparagraph}[1]{\oldsubparagraph{#1}\mbox{}}
\fi

%%% Use protect on footnotes to avoid problems with footnotes in titles
\let\rmarkdownfootnote\footnote%
\def\footnote{\protect\rmarkdownfootnote}

%%% Change title format to be more compact
\usepackage{titling}

% Create subtitle command for use in maketitle
\newcommand{\subtitle}[1]{
  \posttitle{
    \begin{center}\large#1\end{center}
    }
}

\setlength{\droptitle}{-2em}

  \title{}
    \pretitle{\vspace{\droptitle}}
  \posttitle{}
    \author{}
    \preauthor{}\postauthor{}
    \date{}
    \predate{}\postdate{}
  
\usepackage{booktabs}
\usepackage{longtable}
\usepackage{array}
\usepackage{multirow}
\usepackage[table]{xcolor}
\usepackage{wrapfig}
\usepackage{float}
\usepackage{colortbl}
\usepackage{pdflscape}
\usepackage{tabu}
\usepackage{threeparttable}
\usepackage{threeparttablex}
\usepackage[normalem]{ulem}
\usepackage{makecell}

\begin{document}

En primer lugar se elabora un exponential smoothing

La variable respuesta se define como:

\[ Target_i =\begin{cases}Up & Close_{i+d} > Close_{i} \\Down & Close_{i+d} < Close_{i}\end{cases} \]
dónde:

\begin{itemize}
\tightlist
\item
  Close\_i es el precio de cierre del activo en el día ``i''. En este
  trabajo se definen 3 d, que equivalen a las predicciones a 1, 2 y 3
  meses vista, respectivamente. Esta característica se detalla en
  profundidad en la página \_\_ del presente trabajo
\end{itemize}

\begin{Shaded}
\begin{Highlighting}[]
\NormalTok{knitr}\OperatorTok{::}\KeywordTok{kable}\NormalTok{(}\KeywordTok{data.frame}\NormalTok{(}\DataTypeTok{d1=}\DecValTok{20}\NormalTok{,}
                        \DataTypeTok{d2=}\DecValTok{40}\NormalTok{,}
                        \DataTypeTok{d3=}\DecValTok{60}\NormalTok{),}\DataTypeTok{format =} \StringTok{"latex"}\NormalTok{)}
\end{Highlighting}
\end{Shaded}

\begin{tabular}{r|r|r}
\hline
d1 & d2 & d3\\
\hline
20 & 40 & 60\\
\hline
\end{tabular}

\justifying
A causa de la naturaleza inestable de los precios de cierre, la
probabilidad de que un stock tome el mismo valor al cabo de un mes es
ínfima. Por eso la desigualdad no incluye la igualdad.

\begin{Shaded}
\begin{Highlighting}[]
\KeywordTok{library}\NormalTok{(quantmod)}
\end{Highlighting}
\end{Shaded}

\begin{verbatim}
## Loading required package: xts
\end{verbatim}

\begin{verbatim}
## Loading required package: zoo
\end{verbatim}

\begin{verbatim}
## 
## Attaching package: 'zoo'
\end{verbatim}

\begin{verbatim}
## The following objects are masked from 'package:base':
## 
##     as.Date, as.Date.numeric
\end{verbatim}

\begin{verbatim}
## Loading required package: TTR
\end{verbatim}

\begin{verbatim}
## Version 0.4-0 included new data defaults. See ?getSymbols.
\end{verbatim}

\begin{Shaded}
\begin{Highlighting}[]
\KeywordTok{library}\NormalTok{(dplyr)}
\end{Highlighting}
\end{Shaded}

\begin{verbatim}
## 
## Attaching package: 'dplyr'
\end{verbatim}

\begin{verbatim}
## The following objects are masked from 'package:xts':
## 
##     first, last
\end{verbatim}

\begin{verbatim}
## The following objects are masked from 'package:stats':
## 
##     filter, lag
\end{verbatim}

\begin{verbatim}
## The following objects are masked from 'package:lubridate':
## 
##     intersect, setdiff, union
\end{verbatim}

\begin{verbatim}
## The following objects are masked from 'package:base':
## 
##     intersect, setdiff, setequal, union
\end{verbatim}

\begin{Shaded}
\begin{Highlighting}[]
\KeywordTok{library}\NormalTok{(ggplot2)}
\KeywordTok{library}\NormalTok{(kableExtra)}

\KeywordTok{source}\NormalTok{(}\StringTok{"C:/Users/i0386388/Desktop/TFG-master/functions/smooth_target.R"}\NormalTok{)}
\end{Highlighting}
\end{Shaded}

\begin{Shaded}
\begin{Highlighting}[]
\KeywordTok{getSymbols}\NormalTok{(}\DataTypeTok{Symbols =} \StringTok{"KO"}\NormalTok{, }\DataTypeTok{from =}\StringTok{"2000-01-01"}\NormalTok{)}
\end{Highlighting}
\end{Shaded}

\begin{verbatim}
## 'getSymbols' currently uses auto.assign=TRUE by default, but will
## use auto.assign=FALSE in 0.5-0. You will still be able to use
## 'loadSymbols' to automatically load data. getOption("getSymbols.env")
## and getOption("getSymbols.auto.assign") will still be checked for
## alternate defaults.
## 
## This message is shown once per session and may be disabled by setting 
## options("getSymbols.warning4.0"=FALSE). See ?getSymbols for details.
\end{verbatim}

\begin{verbatim}
## 
## WARNING: There have been significant changes to Yahoo Finance data.
## Please see the Warning section of '?getSymbols.yahoo' for details.
## 
## This message is shown once per session and may be disabled by setting
## options("getSymbols.yahoo.warning"=FALSE).
\end{verbatim}

\begin{verbatim}
## [1] "KO"
\end{verbatim}

\begin{Shaded}
\begin{Highlighting}[]
\NormalTok{KO_}\DecValTok{30}\NormalTok{<-}\KeywordTok{smoothing_finance}\NormalTok{(KO,}\DecValTok{30}\NormalTok{)}
\NormalTok{KO_}\DecValTok{60}\NormalTok{<-}\KeywordTok{smoothing_finance}\NormalTok{(KO,}\DecValTok{60}\NormalTok{)}
\NormalTok{KO_}\DecValTok{90}\NormalTok{<-}\KeywordTok{smoothing_finance}\NormalTok{(KO,}\DecValTok{90}\NormalTok{)}
\end{Highlighting}
\end{Shaded}

\begin{Shaded}
\begin{Highlighting}[]
\KeywordTok{chartSeries}\NormalTok{(KO)}
\end{Highlighting}
\end{Shaded}

\includegraphics{smoothing_files/figure-latex/unnamed-chunk-4-1.pdf}

\begin{Shaded}
\begin{Highlighting}[]
\KeywordTok{ggplot}\NormalTok{()}\OperatorTok{+}\KeywordTok{geom_line}\NormalTok{(}\DataTypeTok{data=}\NormalTok{KO_}\DecValTok{30}\NormalTok{,}\KeywordTok{aes}\NormalTok{(}\DataTypeTok{x=}\KeywordTok{as.numeric}\NormalTok{(}\KeywordTok{rownames}\NormalTok{(KO_}\DecValTok{30}\NormalTok{)),}\DataTypeTok{y=}\NormalTok{Close))}
\end{Highlighting}
\end{Shaded}

\includegraphics{smoothing_files/figure-latex/unnamed-chunk-4-2.pdf}

\begin{Shaded}
\begin{Highlighting}[]
\CommentTok{#target variable}
\NormalTok{KO_30_1m<-KO_}\DecValTok{30} \OperatorTok
\StringTok{  }\KeywordTok{slice}\NormalTok{(}\DecValTok{1}\OperatorTok{:}\NormalTok{(}\KeywordTok{nrow}\NormalTok{(.)}\OperatorTok{-}\DecValTok{20}\NormalTok{)) }\OperatorTok\StringTok{ }
\StringTok{  }\KeywordTok{bind_cols}\NormalTok{(}\DataTypeTok{target=}\KeywordTok{target_calc}\NormalTok{(KO_}\DecValTok{30}\NormalTok{,}\DecValTok{20}\NormalTok{))}

\NormalTok{KO_30_2m<-KO_}\DecValTok{30} \OperatorTok
\StringTok{  }\KeywordTok{slice}\NormalTok{(}\DecValTok{1}\OperatorTok{:}\NormalTok{(}\KeywordTok{nrow}\NormalTok{(.)}\OperatorTok{-}\DecValTok{40}\NormalTok{)) }\OperatorTok\StringTok{ }
\StringTok{  }\KeywordTok{bind_cols}\NormalTok{(}\DataTypeTok{target=}\KeywordTok{target_calc}\NormalTok{(KO_}\DecValTok{30}\NormalTok{,}\DecValTok{40}\NormalTok{))}

\NormalTok{KO_30_3m<-KO_}\DecValTok{30} \OperatorTok
\StringTok{  }\KeywordTok{slice}\NormalTok{(}\DecValTok{1}\OperatorTok{:}\NormalTok{(}\KeywordTok{nrow}\NormalTok{(.)}\OperatorTok{-}\DecValTok{60}\NormalTok{)) }\OperatorTok\StringTok{ }
\StringTok{  }\KeywordTok{bind_cols}\NormalTok{(}\DataTypeTok{target=}\KeywordTok{target_calc}\NormalTok{(KO_}\DecValTok{30}\NormalTok{,}\DecValTok{60}\NormalTok{))}
\NormalTok{#######################}

\NormalTok{KO_60_1m<-KO_}\DecValTok{60} \OperatorTok
\StringTok{  }\KeywordTok{slice}\NormalTok{(}\DecValTok{1}\OperatorTok{:}\NormalTok{(}\KeywordTok{nrow}\NormalTok{(.)}\OperatorTok{-}\DecValTok{20}\NormalTok{)) }\OperatorTok\StringTok{ }
\StringTok{  }\KeywordTok{bind_cols}\NormalTok{(}\DataTypeTok{target=}\KeywordTok{target_calc}\NormalTok{(KO_}\DecValTok{60}\NormalTok{,}\DecValTok{20}\NormalTok{))}

\NormalTok{KO_60_2m<-KO_}\DecValTok{60} \OperatorTok
\StringTok{  }\KeywordTok{slice}\NormalTok{(}\DecValTok{1}\OperatorTok{:}\NormalTok{(}\KeywordTok{nrow}\NormalTok{(.)}\OperatorTok{-}\DecValTok{40}\NormalTok{)) }\OperatorTok\StringTok{ }
\StringTok{  }\KeywordTok{bind_cols}\NormalTok{(}\DataTypeTok{target=}\KeywordTok{target_calc}\NormalTok{(KO_}\DecValTok{60}\NormalTok{,}\DecValTok{40}\NormalTok{))}


\NormalTok{KO_60_3m<-KO_}\DecValTok{60} \OperatorTok
\StringTok{  }\KeywordTok{slice}\NormalTok{(}\DecValTok{1}\OperatorTok{:}\NormalTok{(}\KeywordTok{nrow}\NormalTok{(.)}\OperatorTok{-}\DecValTok{60}\NormalTok{)) }\OperatorTok\StringTok{ }
\StringTok{  }\KeywordTok{bind_cols}\NormalTok{(}\DataTypeTok{target=}\KeywordTok{target_calc}\NormalTok{(KO_}\DecValTok{60}\NormalTok{,}\DecValTok{60}\NormalTok{))}
\NormalTok{##############################################3}
\NormalTok{KO_90_1m<-KO_}\DecValTok{90} \OperatorTok
\StringTok{  }\KeywordTok{slice}\NormalTok{(}\DecValTok{1}\OperatorTok{:}\NormalTok{(}\KeywordTok{nrow}\NormalTok{(.)}\OperatorTok{-}\DecValTok{20}\NormalTok{)) }\OperatorTok\StringTok{ }
\StringTok{  }\KeywordTok{bind_cols}\NormalTok{(}\DataTypeTok{target=}\KeywordTok{target_calc}\NormalTok{(KO_}\DecValTok{90}\NormalTok{,}\DecValTok{20}\NormalTok{))}

\NormalTok{KO_90_2m<-KO_}\DecValTok{90} \OperatorTok
\StringTok{  }\KeywordTok{slice}\NormalTok{(}\DecValTok{1}\OperatorTok{:}\NormalTok{(}\KeywordTok{nrow}\NormalTok{(.)}\OperatorTok{-}\DecValTok{40}\NormalTok{)) }\OperatorTok\StringTok{ }
\StringTok{  }\KeywordTok{bind_cols}\NormalTok{(}\DataTypeTok{target=}\KeywordTok{target_calc}\NormalTok{(KO_}\DecValTok{90}\NormalTok{,}\DecValTok{40}\NormalTok{))}

\NormalTok{KO_90_3m<-KO_}\DecValTok{90} \OperatorTok
\StringTok{  }\KeywordTok{slice}\NormalTok{(}\DecValTok{1}\OperatorTok{:}\NormalTok{(}\KeywordTok{nrow}\NormalTok{(.)}\OperatorTok{-}\DecValTok{60}\NormalTok{)) }\OperatorTok\StringTok{ }
\StringTok{  }\KeywordTok{bind_cols}\NormalTok{(}\DataTypeTok{target=}\KeywordTok{target_calc}\NormalTok{(KO_}\DecValTok{90}\NormalTok{,}\DecValTok{60}\NormalTok{))}
\end{Highlighting}
\end{Shaded}

De este modo se consigue una tabla 3x3 de combinaciones posibles para la
definición del modelo entre el periodo de alisado exponencial y el
numero de días en el futuro sobre el que se quiere predecir.

\begin{Shaded}
\begin{Highlighting}[]
\NormalTok{a<-}\KeywordTok{data.frame}\NormalTok{(}\DataTypeTok{EMA30=}\KeywordTok{c}\NormalTok{(}\StringTok{"EMA30 & pred1"}\NormalTok{,}\StringTok{"EMA30 & pred2"}\NormalTok{,}\StringTok{"EMA30 & pred3"}\NormalTok{),}
           \DataTypeTok{EMA60=}\KeywordTok{c}\NormalTok{(}\StringTok{"EMA60 & pred1"}\NormalTok{,}\StringTok{"EMA60 & pred2"}\NormalTok{,}\StringTok{"EMA60 & pred3"}\NormalTok{),}
           \DataTypeTok{EMA90=}\KeywordTok{c}\NormalTok{(}\StringTok{"EMA90 & pred1"}\NormalTok{,}\StringTok{"EMA90 & pred2"}\NormalTok{,}\StringTok{"EMA90 & pred3"}\NormalTok{))}
\KeywordTok{rownames}\NormalTok{(a)<-}\KeywordTok{c}\NormalTok{(}\StringTok{"predicción 1 mes"}\NormalTok{,}\StringTok{"predicción 2 meses"}\NormalTok{,}\StringTok{"predicción 3 meses"}\NormalTok{)}

\NormalTok{a<-}\KeywordTok{data.frame}\NormalTok{(}\DataTypeTok{EMA30=}\KeywordTok{c}\NormalTok{(}\DataTypeTok{Res1=}\KeywordTok{table}\NormalTok{(KO_30_1m}\OperatorTok{$}\NormalTok{target),}\DataTypeTok{Res2=}\KeywordTok{table}\NormalTok{(KO_30_2m}\OperatorTok{$}\NormalTok{target),}\DataTypeTok{Res3=}\KeywordTok{table}\NormalTok{(KO_30_3m}\OperatorTok{$}\NormalTok{target)),}
           \DataTypeTok{EMA60=}\KeywordTok{c}\NormalTok{(}\DataTypeTok{Res1=}\KeywordTok{table}\NormalTok{(KO_60_1m}\OperatorTok{$}\NormalTok{target),}\DataTypeTok{Res2=}\KeywordTok{table}\NormalTok{(KO_60_2m}\OperatorTok{$}\NormalTok{target),}\DataTypeTok{Res3=}\KeywordTok{table}\NormalTok{(KO_60_3m}\OperatorTok{$}\NormalTok{target)),}
           \DataTypeTok{EMA90=}\KeywordTok{c}\NormalTok{(}\DataTypeTok{Res1=}\KeywordTok{table}\NormalTok{(KO_90_1m}\OperatorTok{$}\NormalTok{target),}\DataTypeTok{Res2=}\KeywordTok{table}\NormalTok{(KO_90_2m}\OperatorTok{$}\NormalTok{target),}\DataTypeTok{Res3=}\KeywordTok{table}\NormalTok{(KO_90_3m}\OperatorTok{$}\NormalTok{target))) }\OperatorTok\StringTok{ }
\StringTok{  }\KeywordTok{t}\NormalTok{(.)}



\KeywordTok{kable}\NormalTok{(a, }\StringTok{"latex"}\NormalTok{) }\OperatorTok
\KeywordTok{add_header_above}\NormalTok{(}\KeywordTok{c}\NormalTok{(}\StringTok{" "}\NormalTok{, }\StringTok{"Predicción 1 mes"}\NormalTok{ =}\StringTok{ }\DecValTok{2}\NormalTok{, }\StringTok{"Predicción 2 meses"}\NormalTok{ =}\StringTok{ }\DecValTok{2}\NormalTok{,}\StringTok{"Predicción 3 meses"}\NormalTok{=}\DecValTok{2}\NormalTok{)) }\OperatorTok
\KeywordTok{kable_styling}\NormalTok{(}\DataTypeTok{latex_options =} \KeywordTok{c}\NormalTok{(}\StringTok{"striped"}\NormalTok{),}\DataTypeTok{font_size =} \DecValTok{10}\NormalTok{,}\DataTypeTok{bootstrap_options =} \StringTok{"hover"}\NormalTok{)}
\end{Highlighting}
\end{Shaded}

\begin{table}[H]
\centering\begingroup\fontsize{10}{12}\selectfont
\rowcolors{2}{gray!6}{white}

\begin{tabular}{l|r|r|r|r|r|r}
\hiderowcolors
\hline
\multicolumn{1}{c|}{ } & \multicolumn{2}{|c|}{Predicción 1 mes} & \multicolumn{2}{|c|}{Predicción 2 meses} & \multicolumn{2}{|c}{Predicción 3 meses} \\
\cline{2-3} \cline{4-5} \cline{6-7}
  & Res1.Down & Res1.Up & Res2.Down & Res2.Up & Res3.Down & Res3.Up\\
\hline
\showrowcolors
EMA30 & 2142 & 2546 & 2028 & 2640 & 1927 & 2721\\
\hline
EMA60 & 2023 & 2635 & 1895 & 2743 & 1815 & 2803\\
\hline
EMA90 & 1889 & 2739 & 1792 & 2816 & 1820 & 2768\\
\hline
\end{tabular}
\rowcolors{2}{white}{white}\endgroup{}
\end{table}

\captionof{table}{Tabla de frecuencias de las distintas combinaciones de alisado en los datos y definición de la variable respuesta}


\end{document}
